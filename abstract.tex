% !TEX root = ../main.tex

\pagestyle{empty}

\chapter{TL;DR}
\label{abstract}

{\Large \textbf{Algorithmic Meta-Creativity}} --- Fania Raczinski --- Abstract\footnote{``Too long; didn't read''}

\vspace{0.2cm}

Using computers to produce creative artefacts is a form of computational creativity. Using creative techniques computationally is creative computing. \ac{AMC} spans the two---whether this is to achieve a creative or non-creative output. Creativity in humans needs to be interpreted differently to machines. Humans and machines differ in many ways, we have different `brains/memory', `thinking processes/software' and `bodies/hardware'. Often creative output by machines is judged in human terms. Computers which are truly artificially intelligent might be capable of true artificial creativity. Until then, they are (philosophical) zombie robots: machines that behave like humans but aren't conscious. The only alternative is to see any computer creativity as a direct or indirect expression of human creativity using digital means and evaluate it as such. \ac{AMC} is neither machine creativity nor human creativity---it is both. By acknowledging the undeniable link between computer creativity and its human influence (the machine is just a tool for the human) we enter a new realm of thought. How is \ac{AMC} defined and evaluated? This thesis addresses this issue. First \ac{AMC} is embodied in an artefact (a pataphysical search tool: \url{pata.physics.wtf}) and then a theoretical framework to help interpret and evaluate such products of \ac{AMC} is explained.

\textit{\textbf{Keywords:} Algorithmic Meta-Creativity, Creative computing, Pataphysics, Computational Creativity, Creativity}

\cleardoublepage
