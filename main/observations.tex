% !TEX root = ../main.tex

\chapter{Observations}
\label{ch:observations}

\startcontents[chapters]

\vfill

\begin{alltt}\sffamily
Paying no attention to his fellow mites,
mérite pas que vous fassiez attention à moi,
and told him to look after a calf she had bought,
and whilst he was looking at it attentively.

Phedon the fact affirm'd,
comment peux,
ne faites aucune attention à mon air,
in fact.

For sure Ulysses in your look appears,
was nearly out of her mind,
I omitted none of the common forms attending a royal audience.

And the consequences attending thereupon,
impotent of mind,
shape at the moment of looking at the time.
\end{alltt}

\newpage
\minicontents
\spirals

\todo{summarise thesis, contributions etc. conclude by comparing against introduction}

a wide range of subject areas such as computer science, psychology, linguistics, literature, art and poetry, languages and mathematics.
\todo{refer back to these in conclusion}


\section{Outroduction}

The last XYZ chapters have explained in probably too much detail what \ac{AMC} is and how to evaluate it. Given that this spans so many different disciplines the contextual background information necessary to understand the research was presented in a broad literature survey in chap XYZ. This also posed a problem for choosing the right methodology for the project. In the end a transdisciplinary approach was chosen as described in chap XYZ with a heavy component of iterative exploratory rapid-prototyping to develop an artefact to demonstrate what \ac{AMC} is. 

This artefact is presented on \url{pata.physics.wtf}. It is an artwork dedicated to \ac{AMC}, pataphysics, \ac{OULIPO} and programming culture.

A critique of computer creativity and its current evaluation formed the starting point for a new framework which was introduced in chap XYZ. The general conlcusion of the thesis was made up of the critical analysis and further work chapters as awell as this final concluding chapter right at the end.

The appendix contains various code snippets and peripheral pieces vaguely related or relevant for parts of this thesis. The code of the website is included on a CD CHECK attached to the back of the front cover. Of course the website is also available online at \url{pata.physics.wtf}.

\todo{check if i need to submit a CD?}


\section{Issues}

\begin{itemize}
  \item Summarise issues in analysis Section
  \item summarize future work
\end{itemize}


\section{Answers}
\label{s:answers}

In the introduction I asked several questions that I attempted to asnwer with my research. This section contains brief answers from 50.000 feet\footnote{Inspired by Time Berners-Lee's articles on the Web in 1998---/url{http://www.w3.org/DesignIssues/Architecture.html}}, meaning they provide a top-down view of the answer and pointers to where in the thesis readers can find more elaborations.

\todo{add chapter references}
\todo{these are closed questions, not research questions}
\begin{description}
  \item[Can computers or algorithms be considered creative?] In short: no. In chapters~\ref{ch:evaluation}\marginnote{§~\ref{ch:evaluation}} and~\ref{ch:interpretation}\marginnote{§~\ref{ch:interpretation}} I have gone into great detail of why I believe that this cannot happen any time soon (see argument of zombies). They can be `creative' (adj/adv CHECK) but the source of the creativity is the programmer of the machine not the machine itself.
  \item[Can pataphysics facilitate creativity?] Yes. Pataphysics provides many principles which can be turned into technicques and constraints which is well known to be able to support creativity (see chapter~\ref{ch:foundations}\marginnote{§~\ref{ch:foundations}}). This is also evident in the \ac{OULIPO} and their use of constraints (see chapter~\ref{ch:creativity}\marginnote{§~\ref{ch:creativity}}).
  \item[Can a creative process be automated or emulated by a computer?] Yes, in theory. It mainly depends how you define the creative process and that is fairly subjective. See more in chapter~\ref{ch:creativity}\marginnote{§~\ref{ch:creativity}} and~\ref{ch:interpretation}\marginnote{§~\ref{ch:interpretation}}.
  \item[Can human and computer creativity be objectively measured?] No. As discussed in chapter~\ref{ch:interpretation}\marginnote{§~\ref{ch:interpretation}} since the perception of creativity is subjective it cannot be quantified in objective terms. By providing a framework\marginnote{§~\ref{s:oec}} that takes into account all possible contextually relevant contributors though we can approximate an objective evaluation.
  \item[Can information retrieval be creative?] Yes. There are many ways this can be achieved too as mentioned in chapter~\ref{ch:analysis}\marginnote{§~\ref{ch:analysis}}.
  \item[Can search results be creative rather than relevant?] Yes, although this is also subjective. What is creative to some might not be creative to everybody. The artefact also nicely showed the difference in perception of results simply based on design of the content (see chapter~\ref{ch:analysis}\marginnote{§~\ref{ch:analysis}}).
\end{description}



\section{Contributions}

mention to whom these could be useful

\todo{write more}

This doctoral project can be broken down into four main contributions.

\begin{itemize}
  \item Three pataphysical search algorithms (clinamen, syzygy and antinomy).
  \item A creative exploratory search tool demonstrating the algorithms in the form of a website \url{http://pata.physics.wtf}.
  \item A set of subjective parameters for defining creativity.
  \item An objective framework for evaluating creativity.
\end{itemize}

In a more practical sense this project has spawned several publications, talks and exhibitions (a full list is in preface~\ref{ch:pubs}\marginnote{sec~\ref{ch:pubs}}). Further talks were given by Andrew Hugill at various conferences and events throughout the world where he mentioned my work. My publications were cited in other academic publications and my website was mentioned on Reddit\footnote{Although absolutely nobody seemed interested in it. No idea who posted it or how he found it.}. My job here is done.


\section{And Finally}

\emph{Pataphysics is the science\ldots}

\stopcontents[chapters]
