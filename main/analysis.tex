% !TEX root = ../main.tex

\chapter{Patanalysis}
\label{ch:analysis}

\startcontents[chapters]

\vfill

Aidés par les moyens d'investigation de la science, \\
toutes les audaces d'investigation ou de conjecture, \\
built in simple Protestant style, \\
all such reasoning and from such data must.

And I style him friend, \\
its whole style differed materially from that of Legrand, \\
the calculus of Probabilities, \\
n'échappaient à leur investigation.

Another line of reasoning partially decided me, \\
to make an anatomical dissection of its body and, \\
ce style en débâcle et innavigable.

In a style Of gold, \\
que la sobriété du style se conduit de la sorte, \\
still a point worthy very serious investigation.

\newpage
\minicontents
\spirals


CRITICAL ANALYSIS!

\todo{queneau - how many results do i generate}
\todo{how does my poetry relate or contribute}


\section{technical analysis}
\begin{draft}
  is it fit for purpose? robust? as planned? \\
  do the algorithms fulfil their ideological purpose? \\

\end{draft}

\section{creative analysis}
\begin{draft}
  oulipo? \\
  literary deconstruction and recombining to make new creative output? \\
  perception of results (poetry, source, algorithm) \\
  discuss applications from before (stimulates creative detour away from the obvious) \\

\end{draft}


\section{Problems}

\todo{discuss problems with algorithms, pros and cons...}
This function exhibits the same problem as mentioned above for the syzygy, just much worse. Arguably, some words just do not appear to have an opposite, but the pataphysical antinomy should still be able to find a match. A better thesaurus or a larger index (e.g.\ based on more than one book –-- or, of course, the Web) could improve this method.

\begin{itemize}
  \item Antinomy algorithm not producing many results
  \item Image and Video search relying on APIs
  \item Performance slow
  \item
\end{itemize}


\section{Shortcomings}

From here, we can try to implement different algorithms or different pataphysical concepts within our existing tool or built a different system. The next logical step would be to implement a fully functioning Web search engine using the algorithms described in this paper. But before we go into further development, it might be worth evaluating and interpreting the results produced by the prototype.

\begin{itemize}
  \item
  \item
  \item
  \item
  \item
  \item
  \item
  \item
  \item
\end{itemize}


\section{Focus Group}

\todo{TODO:FOCUS2, study on evaluation before and after framework}

\begin{enumerate}
  \item ask people to judge prototype
  \item explain criteria and framework
  \item ask to judge prototype again
  \item compare results
\end{enumerate}







\section{Technical}

\begin{table}[htb]
  \begin{tabu}{X[1,L]|X[3,L]X[3,L]X[2,L]}
  \toprule
  % \cline{2-4}
  &
  \textbf{clinamen}
  &
  \textbf{syzygy}
  &
  \textbf{antinomy}
  \\ \midrule
  \textbf{clear}
  &
  altar, leaf, pleas, cellar
  &
  vanish, allow, bare, pronounce
  &
  opaque
  \\ \midrule
  \textbf{solid}
  &
  sound, valid, solar, slide
  &
  block, form, matter, crystal, powder
  &
  liquid, hollow
  \\ \midrule
  \textbf{books}
  &
  boot, bones, hooks, rocks, banks
  &
  dialogue, authority, record, fact
  &
  ---
  \\ \midrule
  \textbf{troll}
  &
  grill, role, tell
  &
  wheel, roll, mouth, speak
  &
  ---
  \\ \midrule
  \textbf{live}
  &
  love, lies, river, wave, size, bite
  &
  breathe, people, domicile, taste, see, be
  &
  recorded, dead
  \\ \bottomrule
  \end{tabu}
\caption[Comparison of algorithms]{Comparison of algorithms}
\label{algorithmscomp}
\end{table}


\section{Personal}

\begin{itemize}
  \item Illness
  \item Real Life
  \item James?
  \item
\end{itemize}


\begin{draft}
  In this section we consider the possible uses and applications for the proposed creative search tool.

  Our target audience is not quite as broad as that of a general search engine like Google. Instead, we aim to specifically cater for users who can appreciate creativity or users in need of creative inspiration. Users should generally be educated about the purpose of the search tool so that are not discouraged by what might appear to be nonsensical results. Users could include artists, writers or poets but equally anybody who is looking for out-of-the-box inspirations or simply a refreshingly different search engine to the standard.

  The way we display and label results produced by the tool can influence how the user perceives them. The current prototype for example separates the results into its three components but we could have equally just mixed them all together. The less transparent the processes in the background (e.g.\ which algorithm was used, how does the result relate to the query precisely, etc.) are for the user, the more difficult it might be to appreciate the search.

  There are many ways a pataphysical search tool could be used across disciplines.

  In literature, for example, it could be used to write or generate poetry, either practically or as a simple aid for inspiration. We are not limited to poetry either; novels, librettos or plays could benefit from such pataphysicalised inspirations. One can imagine tools using this technology that let you explore books in a different ordering of sentences (a sort of pataphysical journey of paragraph hopping), tools that re-write poems or mix and match them together. Even our simple prototype shows potential in this area and could be even more powerful if we extended it to include more base texts, for example the whole set of books contained in Faustroll’s library ([20] and also [12]). A richer body of texts (by different authors) would produce a larger index which would possibly find many more matches through WordNet and end in a more varied list of results.

  From a computer science perspective it could be used as one of the many algorithms used by traditional search engines for purposes like query feedback or expansion (e.g. “did you mean … “or “you might also be interested in … “). Depending on how creative we want the search engine to be, the higher we would rank the importance of this particular algorithm. One of the concepts related to the search tool, namely patadata, could have an impact on the development of the Semantic Web. Just as the Semantic Web is about organizing information semantically through objective metadata, patadata could be used to organize information pataphysically in a subjective way.

  The prototype tool is already being used in the creation of an online opera, provisionally entitled from [place] to [place], created in collaboration with The Opera Group, an award-winning, nationally and internationally renowned opera company, specialising in commissioning and producing new operas. In particular, it is being used to create the libretto for one of the virtual islands whose navigation provides the central storyline for the opera. The opera will premiere in 2013, and will continue to develop thereafter, deploying new versions of the tool as they appear.
\end{draft}






\stopcontents[chapters]
