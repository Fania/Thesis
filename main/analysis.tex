% !TEX root = ../main.tex

\chapter{Patanalysis}
\label{ch:analysis}

\startcontents[chapters]

\vfill

Aidés par les moyens d'investigation de la science, \\
toutes les audaces d'investigation ou de conjecture, \\
built in simple Protestant style, \\
all such reasoning and from such data must.

And I style him friend, \\
its whole style differed materially from that of Legrand, \\
the calculus of Probabilities, \\
n'échappaient à leur investigation.

Another line of reasoning partially decided me, \\
to make an anatomical dissection of its body and, \\
ce style en débâcle et innavigable.

In a style Of gold, \\
que la sobriété du style se conduit de la sorte, \\
still a point worthy very serious investigation.

\newpage
\minicontents
\spirals

\todo{apply my own framework to my own product and discuss results}
\todo{apply my own framework to my own framework and discuss results---recursion}


A lot of the analysis on the more theoretical aspects of this research have been discussed in chapters~\ref{ch:foundations} and \ref{ch:interpretation}. The evaluation here is more concerned with the practical artefact \url{pata.physics.wtf} and its interpretation as well as whether or not it actually achieved some of its goals and was true to its inspirations.

\section{Influences}

Looking back over the inspirations for this project (see chapter XYZ), some of the influences can be clearly seen straight away. Others are intentionally a bit more subtle.

\begin{description}
  \item[Syzygy Surfer] The influence of the Syzygy Surfer cannot be overstated. It forms the immediate predecessor to my research. It should not be forgotten that the authors of the Syzygy Surfer are part of my supervisoy team. This is where the initial ideas for the pataphysical algorithms come from. There are differences as well though. For example pataphors were never implemented even though originally suggested. Also, the concept of patadata was never really conceptualised properly. The idea of using ontologies and semantic web technologies such as \gls{rdf} to develop the system was abandoned early on too.
  \item[Faustroll Library] This fictional library of real books was direct inspiration for the Faustroll corpus used in the text search (see chapter XYZ). I tried my best to complete the library as accurately as I could but some of the texts where unsourceable. As with the original, I included some foreign language texts. Since the results (if the Faustroll corpus is chosen of course) are drawn from any of these texts, the mood and style of language is quite distinct and athmospheric.
  \item[Queneau's 100 thousand million poems] Queneau is another one of the inspirations that became a direct influence. The text search can be displayed as poetry in the same style as Queneau's 100 thousand million poems only in digital form and with a larger set of lines. This means that many moere possible poems can be generated by switching individual lines. The outcome is beautiful. In principle Queneau's book has been digitised before but this is (to the best of my knowledge) the first time it's been taken a step further.
  \item[Celestial Emporium of Benevloent Knoweldge] Borges chinese encyclopedia has been an inspiration right from the start. The subtle humour in it is great. The sort of semantic logic behind it ``a passage in Borges, out of the laughter that shattered, as I read the passage, all the familiar landmarks of my thought - our thought, the thought that bears the stamp of our age and our geography - breaking up all the ordered surfaces and all the planes with which we are accustomed to tame the wild profusion of existing things, and continuing long afterwards to disturb and threaten with collapse our age-old distinction between the Same and the Other'' \autocite{Foucault1966} was modeled through the pataphysical algorithsm.
  \item[Yossarian Lives] This has been interesting to watch but if anything was more of a counter inspiration. An example of what I do not want to do. Their socalled methaphoric seacrh engine is hyped but it is wholly unclear of how their algorithm actually create these methaphors. It is hard to comapre against this as it is so different even though we share some of the same goals or principles.
  \item[Library of Babel] The libaray of babel is a great project which has only indirectly influence my work. The pataphysical elements in it are obvious even though perhaps unconscious. The seriousness with which the library is presented, the pseudo-scientific approach, the vagueness of what's actually behind it. Is it random? Or is it indeed the most giangtic digital library of any book every written or even to be written? The sheer perceived scale of the library was part motivation for calculating the numbers of the generatable poems.
  \item[Oulipo] blah blah blah refer to inspirations chapter \todo{write}
  \item[Zen of Python] This group of inspirations is a bit more genrric and influenced lots of little things throughout the prject. The idea of hiding easter eggs on the site, the deliberate errors, the obfuscation, the humour, the jargonisation, and the art and aesthetics behind it. All of that was influenced by programming culture as listed in this section.
\end{description}


\section{Evaluation}

\todo{im evaluating the website - not the project !!!!!!!}
\todo{change font size, capitalisation and dashes}

The website \url{pata.physics.wtf} is supposed to be an example of \gls{amc}.

It seems appropriate to start the critical evaluation of the artefact created as part of this research project with an application of my own framework as suggested in chapter~\ref{ch:interpretation}. I will do this in two ways. FIrst I will sketch a matrix similar to the one shown in figure XYZ to give an overview of the evaluation at a glance. Second I will explain each point in the matrix in a bit more detail to try and bring across the thoughts triggered by the framework. In the end the decision of whether or not the artefact has `passed' the criteria/threshold for \gls{amc} is subjective. The framework is only a guideline. Of course it should be considered that ideally this process should be done by an external party or a panel of judges rather than the artist herself.

\begin{description}
  \item[WHO?] Myself, the programmer and artist of \url{pata.physics.wtf}.
  \item[Person---Novelty] The person behind \url{pata.physics.wtf} is myself. As the sole developer and designer of the product I was responsible for all decisions and all creative input. At the time, I had never worked with Python before, never heard of Pataphysics, and never create a website of this complexity. Of course I had some familiarity with programming in general and I had interests in arts but overall the majority of subjects relevant for the project were novel to me.
  \item[Person---Quality] The quality of my work could perhaps be measured by the existence of bugs in my code or the beauty of the design. Given that the subject area was mostly novel to me and my previous education didn't fully prepare me for this sort of work, there are surprisingly few problems with the code. \todo{not true} The performance is way too slow, the design is not great and not user friendly enough.
  \item[PERSON - VALUE] The value of myself as the researcher on this project is clear in my background. I brought in a varied background and many interests. I had done Computer Science as an undergraduate degree - which gave me an understanding of code necessary to complete the practical aspect of the project but also some of the more theoretical ideas. Having then done a postgraduate degree in Creative Technologies helped introduce me to interdisciplinary work and allowed me to experiment with my creative side. This was essential for the project at hand. It allowed me to see problems from different perspectives.
  \item[PERSON - PURPOSE] I was chosen for this project presumably because of the skills and interests I had demonstrated in the past. On a more insteresting note perhaps---a website doesn't build itself. I created the backend and frontend all by myself. I created the algorithms which form the core of the website.
  \item[PERSON - SPATIAL] Luckily spatial issues are not much of an issue when it comes to Web development. I could work anywhere with an Internet connection and a laptop or computer at hand. This allowed me to be very flexibel with my location. Another aspect to this was my nationality and upbringing. I am originally from Germany. I grew up near a museum on `Art and Media Technology'\footnote{`ZKM'---see \url{http://on1.zkm.de/zkm/e/}} which got me interested in digital art quite early on in live. Also, my father was an office equipment mechanic and I grew up around computers and have always had a strong interest in Web development.
  \item[PERSON - TEMPORAL] A temporal aspect regarding my person was perhaps the time scale and time management of the project. I studied full time, took a year interruption and more to finish. \todo{update timing} Someone else could have done this faster perhaps. The coding is never finished.
  \item[PERSON - EPHEMERAL] I did not actually apply for this PhD programme but my application was forwarded from another department after an unsuccessful application there. This is quite serendipitous.
\end{description}

\spirals

\begin{description}
  \item[Where?] On the Internet via \url{pata.physics.wtf}.
  \item[PLACE - Novelty] The location of \url{pata.physics.wtf} is online. Other art projects have been put online in the past. This is certainly not new. The \gls{ioct} was already established but Professor Andrew Hugill published his monograph on pataphysics the year I started which meant my research into developing pataphysical algorithms was cutting edge. \todo{check dates} 
  \item[PLACE - QUALITY] The site is hosted on a server provided by `OVH'\footnote{\url{https://www.ovh.co.uk/}}. The cost is reasonable and allows enough freedom to run the Python application which forms the search tool. The speed of the server and security and reliablity is high but out of my control.
  \item[PLACE - VALUE] The site is found through a custom \gls{url} (\url{pata.physics.wtf}) and is findable on google. This was chosen because it is a memorable name and the top level domain name (`.wtf') conveys some of the humour needed to appreciate the project.
  \item[PLACE - PURPOSE] The purpose of putting the project online is of course for users to actually be able to use it whenever and whereever they want. Sticking the search tool on a local machine in a museum space for example would not be very interesting. Of course the project is `interactive' in very simple terms, i.e. the user needs to enter a keyword to trigger the pataphysicalisation and the display of the results and then needs to spend some time reading through them or looking throigh the results.
  \item[PLACE - SPATIAL] The OVH server is hosted in France, although that is not really relevant. It should be accessible from all over the world, unless it gets blocked.
  \item[PLACE - TEMPORAL] The hosting and domain name need renewing each year. Website design goes out of date quickly nowadays, so it may have to be redesigned to stay appealing. Being online, the site is available all day every day, so access is not limited to viewing times in a museum or similar constraints.
  \item[PLACE - EPHEMERAL] N/A
\end{description}

\spirals

\begin{description}
  \item[Why?] To demonstrate pataphysical creative exploratory search algorithms---overall an example of \gls{amc}.
  \item[PURPOSE - Novelty] The concepts behind the search tool are novel. Creative search has been attempted before as discussed in chapter CYZ but not specifically with Pataphysics as its inspiration.
  \item[PURPOSE - QUALITY] Whether or not the use of pataphysics over another creative technique is better can only be determined with further study.
  \item[PURPOSE - VALUE] Having a clear aim is always helpful, and in the case of \url{pata.physics.wtf} that aim pervades the site through and through. The main functionality is to provide creative search not relevant lookup search. The value is subjective to each user.
  \item[PURPOSE - PURPOSE] N/A
  \item[PURPOSE - SPATIAL] The fact that some of the texts in the search results are french or german is a conscious choice not accident or necessity. This language barrier reminds users of language spaces, borders, inaccessibility and originality. It reminds users that some texts may be translated from a different language. Its a sign of equality to include different languages representing different locations. From a different perspective, it was also imperative to make the system available from all over the world. This is also why the site was created to be responsive---to allow users to access it comfortably from their phones, tablets, laptops or desktop computers.
  \item[PURPOSE - TEMPORAL] A similar point is true for the time aspect. The idea was to allow users to access the system anytime.
  \item[PURPOSE - EPHEMERAL] Of course the system may appear serendipitous or random at times but the underlying logic certainly is not random. It was important to bring across a sense of structure in the results and the pataphysical algorithms hopefully achieve that.
\end{description}

\spirals

\begin{description}
  \item[What?] \url{pata.physics.wtf}: an exploratory algorithmic meta-creative search tool.
  \item[Product - Novelty] The actual website itself doesn't use any groundbreakingly new frameworks or techniques other than the patalgorithms described in chapter XYZ. 
  \item[Product - Quality] The website looks polished and functions without major incidents.
  \item[Product - Value] The value of the website is discussed in chapter XYZ and the fact that it has been used to create a libretto for an opera is great.
  \item[Product - Purpose] The purpose was to create an example of \gls{amc}.
  \item[Product - Spatial] 
  \item[Product - Temporal]
  \item[Product - Ephemeral]
\end{description}

\spirals

\begin{description}
  \item[How?] By combining pataphysics with creativity to create patalgotihms.
  \item[PROCESS - NOVELTY] The algorithms are novel. The approach of using pataphysics to inspire the creative element of the project is novel.
  \item[PROCESS - QUALITY] The development process was experimental. It involved a lot of trial and error to get things right.
  \item[PROCESS - VALUE] The algorithms produce interesting results.
  \item[PROCESS - PURPOSE] The algorithms are an example of creative computing using pataphyscs.
  \item[PROCESS - SPATIAL] The algorithms rely on corpora which they need access to, to work properly.
  \item[PROCESS - TEMPORAL] The startup process is long and pataphysicalisation can take some time.
  \item[PROCESS - EPHEMERAL] There is an element of randomness in some of the algorithms, e.g. the image and video search.
\end{description}

What does this description of the 
\todo{create a template matrix to fill in with colours or whatever and then summarise the above items underneath - only highlighting the interesting ones}

\begin{draft}
  What does this now tell us? It shows that we can almost always argue for creative aspects in each of the points raised in the matrix. Is the product fit for purpose? That's subjective but I would argue that yes. Is the product robust and working as planned? Yes, it works reliably and as planned.

  This evaluation is subjective.
\end{draft}


\section{creative analysis}
\begin{draft}
  literary deconstruction and recombining to make new creative output? \\
  perception of results (poetry, source, algorithm) \\
  discuss applications from before (stimulates creative detour away from the obvious) \\

  How does this relate to Oulipo and Pataphysics? 

  Perhaps this is where I should talk a bit about the perception of results in their different output formats/styles. The poetry is automatically read with more gravity. Sorting by sources is a game of exploration or algorithms which becomes a game of finding the similarities within the result sets. They are different ways to view the same things and yet have a drastic influence of how the results are perceived. This also applies to the image and video search. Presenting results in spiral form is weird. Its hard to see where one image ends and another starts, they just kind of blur into each other. When listed as a list they immediately become more boring.

  talk abit about what the original plan was for some of the big changed elements in the website, e.g. the image search running 10 times on different keywords rather than running once with 10 results for the same keyword.
\end{draft}






\section{Problems}

\todo{discuss problems with algorithms, pros and cons...}
This function exhibits the same problem as mentioned above for the syzygy, just much worse. Arguably, some words just do not appear to have an opposite, but the pataphysical antinomy should still be able to find a match. A better thesaurus or a larger index (e.g.\ based on more than one book –-- or, of course, the Web) could improve this method.

\begin{itemize}
  \item
  \item
  \item 
  \item
\end{itemize}


\section{Shortcomings}

From here, we can try to implement different algorithms or different pataphysical concepts within our existing tool or built a different system. The next logical step would be to implement a fully functioning Web search engine using the algorithms described in this paper. But before we go into further development, it might be worth evaluating and interpreting the results produced by the prototype.

\begin{itemize}
  \item
  \item
  \item
  \item
  \item
  \item
  \item
  \item
  \item
\end{itemize}




\section{Patalgorithms}

It is quite interesting to compare the three different algorithms with each other. By removing the clutter (in this case the sentence surrounding the pataphysicalised keyword) we can see a few example results side by side below in table XYZ.

\begin{table}[htb]
  \begin{tabu}{X[1,L]|X[3,L]X[3,L]X[2,L]}
  \toprule
  % \cline{2-4}
  &
  \textbf{clinamen}
  &
  \textbf{syzygy}
  &
  \textbf{antinomy}
  \\ \midrule
  \textbf{clear}
  &
  altar, leaf, pleas, cellar
  &
  vanish, allow, bare, pronounce
  &
  opaque
  \\ \midrule
  \textbf{solid}
  &
  sound, valid, solar, slide
  &
  block, form, matter, crystal, powder
  &
  liquid, hollow
  \\ \midrule
  \textbf{books}
  &
  boot, bones, hooks, rocks, banks
  &
  dialogue, authority, record, fact
  &
  ---
  \\ \midrule
  \textbf{troll}
  &
  grill, role, tell
  &
  wheel, roll, mouth, speak
  &
  ---
  \\ \midrule
  \textbf{live}
  &
  love, lies, river, wave, size, bite
  &
  breathe, people, domicile, taste, see, be
  &
  recorded, dead
  \\ \bottomrule
  \end{tabu}
\caption[Comparison of algorithms]{Comparison of algorithms}
\label{algorithmscomp}
\end{table}

Seeing the results in a table like this gives an almost immediate idea of how each algorithm works. This is not meant to to be transparent and perhaps only after knowing the ins and outs of the algorithms can one recognise how each result was found. The clinamen results show words that contain one or two spelling errors of the original query term. It is perhaps counter intuitive to have words such as `altar', `leaf' and `cellar' be classed as spelling errors of the word `clear' but they clearly could be. Remember that a spelling error can be classed in one of four ways: (1) deletion, (2) insertion, (3) substitution and (4) transposition. So, going from `clear' to `altar' is an instance of two times case 3 (`c' is replace by `a' and `e' is replaced by `t') and going from  `clear' to `leaf' is an example of case 1 (`c' is deleted) and case 3 (`r' is replaced by `f').

Looking at the second column, the syzygy results, shows semantic relationship between the original query term and the results. Again, this may not be immediatly noticeable but certainly once you know how the process works you can recognise the common relations. This is especially evident for the antinomy algorithm.

\spirals

However it is equally interesting to compare some of the sentences



\section{Personal}

On a more personal note, real life made it harder to keep the project on track and on time. Chronic illness, big life changes, dramatic family events and financial difficulties were all happening at once. The research has changed quite a bit since the beginning. That isn't necessarily bad but it costs time. Now the project is mien and aI am proud of it.


\begin{draft}
  DELETE THIS
  In this section we consider the possible uses and applications for the proposed creative search tool.

  Our target audience is not quite as broad as that of a general search engine like Google. Instead, we aim to specifically cater for users who can appreciate creativity or users in need of creative inspiration. Users should generally be educated about the purpose of the search tool so that are not discouraged by what might appear to be nonsensical results. Users could include artists, writers or poets but equally anybody who is looking for out-of-the-box inspirations or simply a refreshingly different search engine to the standard.

  The way we display and label results produced by the tool can influence how the user perceives them. The current prototype for example separates the results into its three components but we could have equally just mixed them all together. The less transparent the processes in the background (e.g.\ which algorithm was used, how does the result relate to the query precisely, etc.) are for the user, the more difficult it might be to appreciate the search.

  There are many ways a pataphysical search tool could be used across disciplines.

  In literature, for example, it could be used to write or generate poetry, either practically or as a simple aid for inspiration. We are not limited to poetry either; novels, librettos or plays could benefit from such pataphysicalised inspirations. One can imagine tools using this technology that let you explore books in a different ordering of sentences (a sort of pataphysical journey of paragraph hopping), tools that re-write poems or mix and match them together. Even our simple prototype shows potential in this area and could be even more powerful if we extended it to include more base texts, for example the whole set of books contained in Faustroll’s library ([20] and also [12]). A richer body of texts (by different authors) would produce a larger index which would possibly find many more matches through WordNet and end in a more varied list of results.

  From a computer science perspective it could be used as one of the many algorithms used by traditional search engines for purposes like query feedback or expansion (e.g. “did you mean … “or “you might also be interested in … “). Depending on how creative we want the search engine to be, the higher we would rank the importance of this particular algorithm. One of the concepts related to the search tool, namely patadata, could have an impact on the development of the Semantic Web. Just as the Semantic Web is about organizing information semantically through objective metadata, patadata could be used to organize information pataphysically in a subjective way.

  The prototype tool is already being used in the creation of an online opera, provisionally entitled from [place] to [place], created in collaboration with The Opera Group, an award-winning, nationally and internationally renowned opera company, specialising in commissioning and producing new operas. In particular, it is being used to create the libretto for one of the virtual islands whose navigation provides the central storyline for the opera. The opera will premiere in 2013, and will continue to develop thereafter, deploying new versions of the tool as they appear.
\end{draft}






\stopcontents[chapters]
