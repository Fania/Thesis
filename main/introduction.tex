% !TEX root = ../main.tex

\chapter{Introduction}
\label{ch:intro}

\emph{Part of this research has been described in a journal article in Digital Creativity in 2013, and I presented a paper at the Creativity and Cognition conference 2013 in Sydney.}

\grule

Imagine a web search engine that does not quite return the results you expect. For example, imagine you search for "animal" and the top three results are a list of animals in the Emperor's possession, followed by instructions about embalming animals and information on a society for animal training. Google's top search results for this query on the other hand return the webpage of an action sports lifestyle brand, the Wikipedia article and a BBC page about animal videos. While there is certainly nothing wrong with Google's results, they are simply not very inspiring. The first example of search results is adapted from Jorge Luis Borges's Chinese Encyclopaedia \citep{Borges2000} \marginpar{§¶} which lists several creative definitions of the term "animal". Whilst they might not provide the kind of information we were initially seeking (if we even had a clear idea of the kind of answers we wanted), they are still perfectly valid results for the query and might even provoke a smirk upon their encounter. These are the kind of search results we are aiming for; provocative, creative, surprising, inspiring and possibly funny yet perfectly valid.

To achieve this sort of creativity in search results we propose the use of pataphysical methods. Pataphysics is highly subjective and particular and is as such very suitable for this kind of transformation from relevant to creative. We hope that the tool will prove useful as a source for information and inspiration and at the same time challenge the way we think about information retrieval on the Web. The Web is not a place limited to one discipline and in fact it has been already suggested to create a transdisciplinary field of ‘Web Science'. Our project will therefore span across several disciplines as well.

\begin{quote}
  "Given the breadth of the Web and its inherently multi-user (social) nature, its science is necessarily interdisciplinary, involving at least mathematics, [computer science], artificial intelligence, sociology, psychology, biology, and economics." \citep{Hendler2008}
\end{quote}

Over the rest of the article, we will examine how pataphysics and creativity map onto one another, give an outline of the field of information retrieval, and discuss how this new type of search could be implemented in future systems. We conclude with a short discussion and summary of the article.

\section{Problem / Motivation / Context}

Jorge Luis Borges has provided us with a very useful example to illustrate our idea. His ‘Chinese Encyclopaedia' \citep{Borges2000} lists the following results under the category of ‘animal'.

\begin{enumerate}
  \item those that belong to the Emperor,
  \item embalmed ones,
  \item those that are trained,
  \item suckling pigs,
  \item mermaids,
  \item fabulous ones,
  \item stray dogs,
  \item those included in the present classification,
  \item those that tremble as if they were mad,
  \item innumerable ones,
  \item those drawn with a very fine camelhair brush,
  \item others,
  \item those that have just broken a flower vase,
  \item those that from a long way off look like flies.
\end{enumerate}

Although these are all perfectly valid results, it is clear that they form a more creative, even poetic, view of what an animal might be than the Oxford English Dictionary's prosaic: "a living organism which feeds on organic matter".

\subsection{Syzygy Surfer}

Pataphysics can provide some useful techniques that are very suitable for creative computing. Hendler and Hugill first suggested the use of three of its principles: clinamen, syzygy and anomaly, in their "Syzygy Surfer".

\begin{quote}
  "The ambiguity of experience is the hallmark of creativity, that is captured in the essence of pataphysics. Traversing the representations of this ambiguity using algorithms inspired by the syzygy, clinamen and anomaly of pataphysics, using a panalogical mechanism applied to metadata, should be able to humanize and even poeticize the experience of searching the Web." \citep{Hendler2013}
\end{quote}

In this article we propose a new type of Web search engine, reminiscent of the experience of ‘surfing the Web'. This is in contrast to current search engines which value relevant results over creative ones. ‘Surfing' used to be a creative interaction between a user and the web of information on the Internet, but the regular use of modern search engines has changed our expectations of this sort of knowledge acquisition. It has drifted away from a learning process by exploring the Web to a straightforward process of information retrieval similar to looking up a word in a dictionary.

Hendler and Hugill introduce a new concept for a web search engine in their paper "The Syzygy Surfer: Creative Technology for the World Wide Web" \citep{Hendler2011}. This PhD research is built on their initial investigation but it should be noted that the Syzygy Surfer is purely conceptual; there is no working prototype available. In the paper, Hendler and Hugill claim that "surfing has become a term for the secure journey to a well-regarded site full of safe content" and suggest a tool that supports "surfing" the web (browse and explore) instead of searching it. Their inspirations come from Borges' "Chinese Encyclopaedia" \citep{Borges2000} (for the underlying poetic sense of unity), Jarry's Pataphysics (for the concept of patadata – data beyond metadata) and  panalogies (parallel analogies – to introduce ambiguity, since it allows various descriptions of the same object) as formulated by Singh \citep{Singh2005}. The search can be executed using three different techniques, using either a syzygy, clinamen or anomaly approach. They also mention the importance of the interface design; features of which should be as follows.

\begin{itemize}
  \item Users should be able to choose the technique for the search (syzygy, clinamen or anomaly)
  \item The system should suggest additional search terms to add to or change the query
  \item The site should have a breadcrumb trail of navigations
  \item The design should be attractive, accessible and adaptable
\end{itemize}

\subsection{Research Questions}

\begin{itemize}
  \item How can we make a search tool that is inspirational rather than informational?
  \item How can we get search results that are unexpected and yet make sense?
  \item How can we rank search results but still be true to Pataphysics philosophy?
  \item How can we represent and structure data to reflect its context, meaning and subjectivity?
  \item How can we present search results in a creative and pataphysical way?
  \item How does Pataphysics relate to creative computing?
  \item How can we use Pataphysics as inspiration for search ranking?
  \item How can we write a specifically creative algorithm?
  \item How can Semantic Web technologies help with the representation of patadata?
  \item What does it mean for search results to be creative/relevant?
\end{itemize}

\section{Aims and Objectives / Methodology}

\begin{quote}
  "Purposive without purpose" Kant
\end{quote}

The aim of this project, in simple words, is to design a tool for creative searching on the Web. It tries to create a fresh way of searching and navigating through information and content on the Internet, to bring back some of the inspirational chance encounters that were so characteristic for libraries. It tries to provide an alternative for or addition to the standard search tools, a different approach that some people might benefit from and others probably won't.

The focus of this project will be on the creation of a unique and innovative ranking algorithm(s) rather than the general architecture of the search engine. These algorithms are in essence what make this tool creative and what drive the general user experience. They will be based on a "patadata" framework and an ontology specific to the knowledge domain of the pseudo-philosophy Pataphysics.

The tool will challenge the way we understand data and metadata and suggest new meanings between them in accordance with the underlying philosophy. The way we think about information is redefined. It will create new standards or rather get rid of all previous standards and classifications. It will create new links and connections between pieces of information that are not simply based on keyword similarities but rather a more poetic sense of unity. It will be a refreshing new view compared to the structured and standardised thinking of computer scientists. It will challenge the binary logic that dominates the world of programmers and concentrate on what they would view as an illogical and unstructured system going against all standards.

The goal is also to investigate creative ways of visually presenting the tool's search results while maintaining the pataphysical philosophy. The project aims to change the way users navigate through their searches as a whole. It aims to turn a simple search into a "journey" that can be traced back to individual steps using breadcrumb trails of navigations. This breadcrumb trail could displays the list of previously visited search results and with such, provide the user with the current location and immediate search history. Being able to go back and forth or adding new constraints to a search at any given point could prove useful and fruitful for the user. It should be transparent how the tool works, how it finds the results, but not overwhelm the user with too much technical information. The user should have a choice of whether or not they want to see it.

This project combines research in science and art. It is an interdisciplinary research project.

This project has roots in disciplines such as Computer Science and Humanities.
\begin{description}
  \item [Information Retrieval]: Software Engineering, Semantic Web
  \item [Pataphysics]: Literature, Philosophy, Ontology
  \item [Creativity]: Cognitive Science, Artificial Intelligence
\end{description}

In regards to my project:
\begin{itemize}
  \item A concept implementation method is used with a descriptive-other approach
  \item A qualitative investigation into if and why the proposed search results are useful will be done
  \item Following experimental methodologies, to evaluate the proposed new solution to the problem of creative search
\end{itemize}

\begin{description}
  \item [Epistemology]: Subjective/Argumentative
  \item [Methodology]: Experimental, Interpretative, Qualitative
  \item [Methods]: Concept implementation, (Heuristic) Evaluation
\end{description}

The main aim of this project is to design a creative search tool based on an innovative ranking system inspired by the idea of Pataphysics. The investigation will specifically focus on the development of a set of search algorithms. Part of this is to develop a new semantic system to describe, compare and order information in the form of "patadata". Next to this, the aim is to investigate creative ways of visually presenting search results while maintaining the underlying pataphysical philosophy. The result will provide users with a unique search" journey" that can be traced back to individual steps using breadcrumb trails of navigations.

\section{Deliverables / Outcomes}

\begin{itemize}
  \item Design a tool for creative searching on the Web
  \item Design pataphysics inspired algorithms to model creativity in this tool
  \item Produce a proof-of-concept prototype
  \item Propose a framework for evaluating and interpreting creative search results
\end{itemize}

The outcome of this project will be a fully functioning search tool for the web. Based on a unique and innovative "ranking" algorithm, the search tool will include a patadata framework and an ontology specific to the knowledge domain of Pataphysics. It will challenge the way we understand data and metadata and suggest new relationships and meanings between them in accordance with the philosophy of Pataphysics.  Research dissemination is a crucial part in any innovative project like this, so an effort will be placed on presenting findings at conferences and publishing papers.


\subsection{Contribution to Knowledge}

Compared to the mainstream search engines available publically, my project's approach to searching will be very different. Google, for example, ranks web pages based on the number and quality of incoming hyperlinks \citep{Google2012}, while I will be using the concept of patadata (data-metadata-patadata) and semantic web technologies.

There have been interdisciplinary projects like Johanna Drucker's Speclab \citep{Drucker2009} which have similar inspirations, and literature on Pataphysics is found in several places \citep{Bok2002, Hugill2012a} although the idea of using it as an inspiration for a search engine appears to be very new \citep{Hendler2013}.

A lot of literature exists on how to model creativity on a computer [5] [7], explaining several theories on how the mind works and how to simulate this on a machine. However none of these have any kind of pataphysical inspiration and aren't applicable to search engines, so we have presented a paper on creativity in search results just recently \citep{Raczinski2013}.

\subsection{Publications}

James Sawle, \textbf{Fania Raczinski} and Hongji Yang (2011) \emph{"A Framework for Creativity in Search Results"}. The 3rd International Conference on Creative Content Technologies, CONTENT'11. Rome, Italy. Pages 54-57.

\noindent Andrew Hugill, Hongji Yang, \textbf{Fania Raczinski} and James Sawle (2013) \emph{"The pataphysics of creativity: developing a tool for creative search"}. Routledge: Digital Creativity, Volume 24, Issue 3. Pages 237-251.

\noindent \textbf{Fania Raczinski}, Hongji Yang and Andrew Hugill (2013) \emph{"Creative Search Using Pataphysics"}. Proceedings of the 9th ACM Conference on Creativity and Cognition, CC'13. Sydney, Australia. Pages 274-280.
