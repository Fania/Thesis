% !TEX root = ../main.tex

\chapter{Introduction}
\label{ch:intro}

\emph{Part of this chapter has been described in a journal article in Digital Creativity in 2013, and I presented a paper at the Creativity and Cognition conference 2013 in Sydney.} \marginpar{§¶}

\grule

{\color{red} HELLO WORLD}% COLOR RED

This thesis describes a general study into computer creativity and in particular a pataphysical methodology applied to creative information retrieval.

A pataphysical methodology for applying creativity to exploratory search.

$\text{A} \ \underset{1}{\text{semantic}} \ \underset{2}{\text{search}} \ \text{tool based on a} \ \underset{3}{\text{pataphysical}} \ \underset{4}{\text{algorithm}} \ \text{using a} \ \underset{5}{\text{'patadata'}} \ \underset{6}{\text{ontology}}$

\begin{enumerate}
  \item Semantic Web technologies
  \item IR techniques
  \item Pataphysics
  \item Ranking algorithms
  \item Data - metdadata - patadata
  \item Ontologies, RDF, OWL, etc
\end{enumerate}

\todo{finish introduction}

In the following chapters, I will examine how pataphysics and creativity map onto one another, give an outline of the field of information retrieval...

\section{Problem / Motivation / Context}

Assumptions: computers can be creative.
\todo{complete assumptions}

From relevant to creative.

Why pataphysics?
Pataphysics is highly subjective and particular and is as such very suitable for this kind of transformation from relevant to creative.
Pataphysics can provide some useful techniques that are very suitable for creative computing.

\begin{quote}
  Purposive without purpose (Kant)
\end{quote}

The aim of this project, in simple words, is to design a tool for creative searching on the Web. It tries to create a fresh way of searching and navigating through information and content on the Internet, to bring back some of the inspirational chance encounters that were so characteristic for libraries.

Current information retrieval systems might be used for creative purposes, however, they do not directly provide creative results to their users, instead they focus on precise and relevant results only. Therefore we argue that a new style of system is required. It is clear that the fundamental problem in this is that standard algorithms are not suited for these problems, with them considering a document to be groupings of words in traditional IR systems, and that an entire document falls under the same classifications in semantic IR systems.

The tool will challenge the way we understand data and metadata and suggest new meanings between them in accordance with the underlying philosophy. The way we think about information is redefined. It will create new standards or rather get rid of all previous standards and classifications. It will create new links and connections between pieces of information that are not simply based on keyword similarities but rather a more poetic sense of unity. It will be a refreshing new view compared to the structured and standardised thinking of computer scientists. It will challenge the binary logic that dominates the world of programmers and concentrate on what they would view as an illogical and unstructured system going against all standards.

Jorge Luis Borges 'Chinese Encyclopaedia' \citep{Borges2000} illustrates this idea very well. The encyclopaedia lists the following results under the category of 'animal'.

\begin{enumerate}
  \item those that belong to the Emperor,
  \item embalmed ones,
  \item those that are trained,
  \item suckling pigs,
  \item mermaids,
  \item fabulous ones,
  \item stray dogs,
  \item those included in the present classification,
  \item those that tremble as if they were mad,
  \item innumerable ones,
  \item those drawn with a very fine camelhair brush,
  \item others,
  \item those that have just broken a flower vase,
  \item those that from a long way off look like flies.
\end{enumerate}

Although these are all perfectly valid results, it is clear that they form a more creative, even poetic, view of what an animal might be than the Oxford English Dictionary's prosaic: ``a living organism which feeds on organic matter''.

We have introduced the motivation and concept for a creative Web search tool and discussed some of the major challenges a project like this faces. With Web search being a major research and learning tool nowadays, it is imperative to think about how such a tool could be (ab)used. Ethical issues that arise through the provision of unexpected results, and the misunderstandings this could lead to, will be discussed in future work. Nevertheless, we believe that creative Web search can facilitate inspirational learning through an exploratory search journey and we hope our tool will provide just that.


\subsection{Related Work}


\subsubsection*{The Syzygy Surfer}

The research presented here is built on the initial ideas of Jim Hendler and Andrew Hugill's "Syzygy Surfer" \citep{Hendler2011, Hendler2013}. They first suggested the use of three pataphysical principles, namely clinamen, syzygy and anomaly, to create a new type of Web search engine, reminiscent of the experience of "surfing the Web". This is in contrast to current search engines which value relevant results over creative ones. "Surfing" used to be a creative interaction between a user and the web of information on the Internet, but the regular use of modern search engines has changed our expectations of this sort of knowledge acquisition. It has drifted away from a learning process by exploring the Web to a straightforward process of information retrieval similar to looking up a word in a dictionary.

\begin{quote}
  The ambiguity of experience is the hallmark of creativity, that is captured in the essence of pataphysics. Traversing the representations of this ambiguity using algorithms inspired by the syzygy, clinamen and anomaly of pataphysics, using a panalogical mechanism applied to metadata, should be able to humanize and even poeticize the experience of searching the Web.\citep{Hendler2013}
\end{quote}

Their inspirations come from Borges' "Chinese Encyclopaedia" \citep{Borges2000} (for the underlying poetic sense of unity), Jarry's Pataphysics \citep{Jarry1996} (for the concept of patadata – data beyond metadata) and  panalogies (parallel analogies – to introduce ambiguity, since it allows various descriptions of the same object) as formulated by Singh \citep{Singh2005}.


\subsubsection*{Yossarianlives}

\begin{quote}
  Traditional search like Google, Bing, Yahoo, or DuckDuckGo returns the most popular results, and gives you expected \& cliché ideas. It's what everyone else already thinks about a topic.

  Yossarian creative search returns diverse and unexpected results that share loose associations to your search. The results help you think about your topic in new ways and generate new ideas.

  If you search "beauty" in Google you get pages and pages of white models. It returns a singular way of thinking about the topic.

  Search "beauty" in Yossarian it returns disparate results with shared attributes, showing many different ways of thinking. Is beauty a diamond (strong, rare, flawless), or is family beautiful (activity, togetherness) or is beauty architected (designed, planned, precision)?\footnote{\url{https://yossarianlives.com/}}
\end{quote}

\begin{quote}
  Use traditional search for when you know what you are looking for.

  Traditional search is for learning what the world already knows.

  Use creative search for when you don't know what you are looking for.

  Creative search is for helping you come up with new ideas.\footnote{\url{https://yossarianlives.com/}}
\end{quote}

\begin{quote}
  Augmented creativity

  Seeing lateral connections is an uniquely human ability, and your individual ability to make these connections is based on your own experience. We build tools that help you see more by suggesting connections that you probably wouldn't have considered otherwise, increasing the diversity and frequency of your ideas, and priming you for creativity.

  Lateral Discovery

  When working to escape filter bubbles too often randomenss and complete seridipity are seen as the answer for new discovery. We build tools that explore lateral association between content and believe this approach can lead to predictable moments of discovery that drive engagement, traffic, and purchase, etc.

  Metaphorical Search

  Search engines today have a problem in that they tell us what the world already knows, reinforcing existing knowledge. We build tools that seek to remedy this by returning results that are disperate, but metaphorically related to the query. These types of results are incredably useful for any one who derives value from new ideas.

  Creative Graph

  Facebook created the Social Graph helping us understand how people and things are connected, Google created the Knowledge Graph to codify the worlds information, and here at Yossarian Lives we are developing the Creative Graph, a radical new way to understand the conceptual relationships between things.\footnote{\url{http://about.yossarianlives.com/index.html}}
\end{quote}


\subsubsection*{The Library of Babel}

\begin{quote}
  The Library of Babel is a place for scholars to do research, for artists and writers to seek inspiration, for anyone with curiosity or a sense of humor to reflect on the weirdness of existence - in short, it’s just like any other library. If completed, it would contain every possible combination of 1,312,000 characters, including lower case letters, space, comma, and period. Thus, it would contain every book that ever has been written, and every book that ever could be - including every play, every song, every scientific paper, every legal decision, every constitution, every piece of scripture, and so on. At present it contains all possible pages of 3200 characters, about 104677 books.

  Since I imagine the question will present itself in some visitors’ minds (a certain amount of distrust of the virtual is inevitable) I’ll head off any doubts: any text you find in any location of the library will be in the same place in perpetuity. We do not simply generate and store books as they are requested - in fact, the storage demands would make that impossible. Every possible permutation of letters is accessible at this very moment in one of the library's books, only awaiting its discovery. We encourage those who find strange concatenations among the variations of letters to write about their discoveries in the forum, so future generations may benefit from their research.\footnote{\url{https://libraryofbabel.info/}}
\end{quote}

\subsection{Research Questions}

\begin{itemize}
  \item How can we make a search tool that is inspirational rather than informational?
  \item How can we get search results that are unexpected and yet make sense?
  \item How can we rank search results but still be true to Pataphysics philosophy?
  \item How can we represent and structure data to reflect its context, meaning and subjectivity?
  \item How can we present search results in a creative and pataphysical way?
  \item How does Pataphysics relate to creative computing?
  \item How can we use Pataphysics as inspiration for search ranking?
  \item How can we write a specifically creative algorithm?
  \item How can Semantic Web technologies help with the representation of patadata?
  \item What does it mean for search results to be creative/relevant?
  \item Can computers be creative?
  \item What does it actually mean to be creative even for a human being, etc
  \item Is pataphysics creative?
  \item What is a relevant search result?
  \item Is creativity irrelevant?
\end{itemize}

\todo{answer research questions in conclusion}

\section{Methodology}

This project combines research in science and art. It is an interdisciplinary research project.

This project has roots in disciplines such as Computer Science and Humanities.
\begin{description}
  \item [Information Retrieval]: Software Engineering, Semantic Web
  \item [Pataphysics]: Literature, Philosophy, Ontology
  \item [Creativity]: Cognitive Science, Artificial Intelligence
\end{description}

In regards to my project:
\begin{itemize}
  \item A concept implementation method is used with a descriptive-other approach
  \item A qualitative investigation into if and why the proposed search results are useful will be done
  \item Following experimental methodologies, to evaluate the proposed new solution to the problem of creative search
\end{itemize}

\begin{description}
  \item [Epistemology]: Subjective/Argumentative
  \item [Methodology]: Experimental, Interpretative, Qualitative
  \item [Methods]: Concept implementation, (Heuristic) Evaluation
\end{description}

\section{Deliverables / Outcomes}

\begin{itemize}
  \item Design a tool for creative searching on the Web
  \item Design pataphysics inspired algorithms to model creativity in this tool
  \item Produce a proof-of-concept prototype
  \item Propose a framework for evaluating and interpreting creative search results
\end{itemize}

\url{http://pata.fania.eu} \marginpar{!!!}

\subsection{Contribution to Knowledge}

Compared to the mainstream search engines available publicly, my project's approach to searching will be very different. Google, for example, ranks web pages based on the number and quality of incoming hyperlinks \citep{Google2012}, while I will be using the concept of patadata (data-metadata-patadata) and semantic web technologies.

There have been interdisciplinary projects like Johanna Drucker's Speclab \citep{Drucker2009} which have similar inspirations, and literature on Pataphysics is found in several places \citep{Bok2002, Hugill2012a} although the idea of using it as an inspiration for a search engine appears to be very new \citep{Hendler2013}.

A lot of literature exists on how to model creativity on a computer [5] [7], explaining several theories on how the mind works and how to simulate this on a machine. However none of these have any kind of pataphysical inspiration and aren't applicable to search engines, so we have presented a paper on creativity in search results just recently \citep{Raczinski2013}.

\subsection{Publications}

James Sawle, \textbf{Fania Raczinski} and Hongji Yang (2011) \emph{"A Framework for Creativity in Search Results"}. The 3rd International Conference on Creative Content Technologies, CONTENT'11. Rome, Italy. Pages 54-57. \citep{Sawle2011}

Andrew Hugill, Hongji Yang, \textbf{Fania Raczinski} and James Sawle (2013) \emph{"The pataphysics of creativity: developing a tool for creative search"}. Routledge: Digital Creativity, Volume 24, Issue 3. Pages 237-251. \citep{Hugill2013d}

\textbf{Fania Raczinski}, Hongji Yang and Andrew Hugill (2013) \emph{"Creative Search Using Pataphysics"}. Proceedings of the 9th ACM Conference on Creativity and Cognition, CC'13. Sydney, Australia. Pages 274-280. \citep{Raczinski2013}

\section{The Hitchhiker's Guide to this Thesis}

\begin{description}
  \item[PREFACE]
  \item[Part I] IN THE BEGINNING...
  \item[Chapter 1] Introduction
  \item[Chapter 2] Methodology
  \item[Part II] IN A GALAXY FAR FAR AWAY...
  \item[Chapter 3] Pataphysics
  \item[Chapter 4] Creativity and Computers
  \item[Chapter 5] IR and NLP
  \item[Part III] THE CORE: TECHNO-LOGIC
  \item[Chapter 6] Theoretical Foundations
  \item[Chapter 7] Practical Implementation
  \item[Chapter 8] Impact and Applications - Case Study
  \item[Part IV] INTECHNOIL-LOGICALYSIS
  \item[Chapter 9] Interpretation / Evaluation
  \item[Chapter 10] Patacritical Analysis
  \item[Part V] HAPPY END
  \item[Chapter 11] Aspirations
  \item[Chapter 12] Observations
  \item[POSTFACE]
\end{description}
