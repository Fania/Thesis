% !TEX root = ../main.tex

\chapter{Introduction}
\label{ch:introduction}

\startcontents[chapters]

\vfill

Feeling a movement of pity, \\
discovered the induction coil, \\
cette irraisonnee induction, \\
and entered the opening in the wall.

Only by some recherche movement, \\
apres coup et sous forme d'introduction, \\
opening his seized manuscript, \\
the enemy made within the enclosure of the vineyard.

Which he had thrown off at the beginning of his labor, \\
in opening so exactly at the, \\
than the thirst of my paternity.

We can then start at once, \\
and whose informing voice had consigned me to the hangman, \\
as any person at all conversant with authorship may satisfy himself at.

\newpage
% {\huge \textbf{Chapter {\color{black} \ref{ch:introduction}}}}
\minicontents
\spirals

This thesis describes \emph{Algorithmic Meta-Creativity}. More precisely it is about using creative computing to achieve computer creativity.

The project is transdisciplinary;\marginnote{§~\ref{ch:methodology}} it is heavily inspired by the absurd french pseudo-philosophy pataphysics\marginnote{§~\ref{ch:pataphysics}} and draws from a wide range of subject areas such as computer science, psychology, linguistics, literature, art and poetry, languages and mathematics.
\todo{refer back to these in conclusion}

The research\marginnote{§~\ref{ch:foundations}} included exploring what it means to be creative as a human, how this translates to machines and how pataphysics relates to creativity.

\todo{tie website together with theory}
\begin{draft}
  
\end{draft}

The outcome\marginnote{§~\ref{ch:implementation}} is presented as a website -\url{pata.physics.wtf}- written in 5 different programming languages\footnote{Python, HTML, CSS, Jinja, JavaScript}, making calls to 6 external Web services\footnote{Microsoft Translate, WordNet, Bing, Getty, Flickr, YouTube}, in a total of over 3000 lines of code\footnote{2864 lines of code, 489 lines of comments - as of 08 Dec 2015} spread over 30 files.
\todo{update these numbers}

The main purpose of the system above is to demonstrate three creative \emph{patalgorithms} in the context of exploratory information retrieval. A browsing rather than a search engine, it presents results in various formats such as sonnets and golden spirals. Immediate inspirations\marginnote{§~\ref{ch:inspirations}} come from fictional character \emph{Doctor Faustroll} created by french absurdist and father of pataphysics Alfred Jarry \citeyear{Jarry1996}, the fantastic taxonomy of the \emph{Celestial Emporium of Benevolent Knowledge} by magical realist Jorge Luis Borges \citeyear{Borges2000} and \emph{A Hundred Thousand Billion Poems} by pataphysician and Oulipo co-founder Raymond Queneau amongst others \citeyear{Queneau1961}.

In a sense the system partially automates the creative process, generating results on demand, which allows users to focus on their own personal artistic evaluation rather than production.

\todo{what is the relationship between pata.physics.wtf and my evaluation framework? - there isn't any really.}

\begin{draft}
  The creative process or problem solving is a move from the abstract to the concrete. Creative evaluation is a move from subjective to objective (defining the subjective criteria for creating a product in terms of objective understanding).
\end{draft}

Another area I explored is the problem of objective evaluation and interpretation\marginnote{§~\ref{ch:interpretation}} of subjective creativity specifically in regards to computers. I argue that the most appropriate way to approach this is by looking at five objective constraints (person, process, product, place, purpose) and seven subjective criteria (novelty, value, quality, purpose, spatial, temporal, ephemeral) holistically and by understanding that humour and art `lie in the ear and eye of the beholder'\ldots

This resulted in an \emph{interpretation framework}\marginnote{§~\ref{s:framework}} visualised as an evaluation matrix (5 constraints x 7 criteria) which can be used to quantitatively and qualitatively measure the creativity of a given artefact (be that man-made or machine-made).


\section{Motivation}

My personal interest in this project comes from a background in computer science and a longstanding interest in art. Most recently I managed to successfully combine my technical skills with my creative side for a Master of Science degree in Creative Technologies at \gls{dmu}\footnote{A passive interactive installation, augmenting a live video stream of users with interactive elements using motion tracking algorithms. See \url{msc.fania.eu}.}. I knew Andrew Hugill through his involvement in the \gls{ioct} at \gls{dmu} and when he pitched his \emph{Syzygy Surfer} \autocite{Hendler2011, Hendler2013} idea to me in an interview, I was immediately drawn in by its underlying sense of humour and the transdisciplinary nature of the project.

\spirals

Computers\marginnote{§~\ref{ch:technology}} are binary machines; the world is black and white to them (0 and 1, on and off). Programmers can run abstract high-level commands which are executed in sequence (fast speed gives the illusion of multitasking). They are precise, structured, logical and generally abide by strict standards. Computers can only be creative if they are given clear instructions as to how. Information retrieval is generally focused on relevance of results in regards to the query.

Pataphysics\marginnote{§~\ref{ch:pataphysics}} emerged during the \emph{Belle Époque}\footnote{1871---1914} in France and has directly or indirectly influenced various artistic movements such as Dada, Symbolism, Surrealism, Oulipo and Absurdist Theatre. Pataphysics is highly subjective and particular, values exceptions, the imaginary and the mutually incompatible.

Creativity\marginnote{§~\ref{ch:creativity}} is often studied at various levels (neurological, cognitive, and holistic/systemic), from different perspectives (subjective and objective) and characteristics (combinational, exploratory and transformative). It is usually defined in terms of value, originality and skill.

Combining computing with pataphysics seems impossible --- although the points below highlight just how intriguing a possible combination of the two would be.

\begin{itemize}
  \item Polymorphism (generalisation) opposes particularity.
  \item Precision opposes exceptions and contradictions.
  \item Logic and structure oppose the imaginary and paradox.
  \item Cross-compatibility opposes the mutually exclusive.
  \item Responsiveness opposes the specific.
  \item Relevance opposes the creative.
\end{itemize}

This apparent dichotomy of computing and pataphysics is alluring. Christian B{\"o}k argued that pataphysics `sets the parameters for the contemporary relationship between science and poetry.' \citeyear{Boek2002} Pataphysics suddenly seems like the perfect choice infusing computers (science) with creativity (poetry).

Combining pataphysics with creativity\marginnote{\faicon{table}~\ref{tab:creatpata}} is easier. The ideas of  combinatorial, exploratory and transformative creativity map quite nicely onto some pataphysical concepts such as clinamen, syzygy, antinomy and anomaly.

\todo{expand here}

\begin{fcom}

\todo{mention that i apply creativity to humans and machines}


Why not just use randomness\footnote{randonmess} you ask? Because there has to be an injection of meaning at some point. Randomness is easy. Andrew Hugill originally suggested that the project should be `purposive without purpose'.

\begin{quotation}
  [\ldots] through aesthetic judgments, beautiful objects appear to be ``purposive without purpose'' (sometimes translated as ``final without end''). An object's purpose is the concept according to which it was made (the concept of a vegetable soup in the mind of the cook, for example); an object is purposive if it appears to have such a purpose; if, in other words, it appears to have been made or designed. But it is part of the experience of beautiful objects, Kant argues, that they should affect us as if they had a purpose, although no particular purpose can be found. \sourceatright{\autocite[ch.2a]{Burnham2015}}
\end{quotation}

pata is purposeless but i use it to give structure
im giving structure to something purposeless
\end{fcom}

\todo{this conflicts with the idea of using pataphysics really over randomness}

\todo{put pointers from intro to the various chapters}

Another motivating factor for this project was the lack of research in the particular area of creative computing\marginnote{§~\ref{ch:creativity}} in general. The discipline of computational creativity has emerged fairly recently\footnote{The first International Conferences on Computational Creativity ran in 2010 for example.} from a background in \gls{ai}. It appears to focus a lot more on the outcome of a product that would be judged creative rather than the actual process. Creative computing focuses on producing creative algorithms which may or may not have creative outputs. This was first addressed in \autocite{Raczinski2013} and later expanded into a definite description of this new discipline \autocite{Hugill2013c}.


\section{Questions}

Research dealing with subjective ideas and concepts like creativity throws up a lot of questions. My intention is to address them all throughout this thesis, although some of them will not have definite binary answers.

\todo{add section refs of answers to each question}
\todo{add more questions}

\begin{itemize}
  \item Can computers or algorithms be considered creative?
  \item Can pataphysics facilitate creativity?
  \item Can a creative process be automated or emulated by a computer?
  \item Can human and computer creativity be objectively measured?
  \item Can information retrieval be creative?
  \item Can search results be creative rather than relevant?
\end{itemize}

\todo{answer research questions in conclusion}

\section{Methodology}
\label{s:intromethod}

This project combines research\marginnote{§~\ref{ch:methodology}} in science and art making it transdisciplinary.

\todo{update from methodology chapter}

\begin{description}
  \item [Pataphysics] Literature, Philosophy
  \item [Creativity] Cognitive Science, \gls{ai}
  \item [Computing] Software Engineering, Information Retrieval, \gls{nlp}
\end{description}

This is practice-based research, meaning that a part of my submission for the degree of Doctor of Philosophy is an artefact demonstrating my original contribution to knowledge. The thesis provides the context of this artefact and critically analyses and discusses the experimental process and outcome.

\todo{remove practice based research stuff}

\begin{description}
  \item [Epistemology] Subjective, Exploratory, Experimental
  \item [Methodology] Practice-Based
  \item [Methods] Creative computing, Web Development, Literature Review
\end{description}

\todo{mention focus group etc}

The general process\marginnote{§~\ref{ch:implementation}} of my project was as follows.

\begin{enumerate}
  \item Conduct extensive literature review into the various subjects involved,
  \item develop pataphysical algorithms,
  \item develop an evaluation framework,
  \item design a system to demonstrate algorithms,
  \item develop a website for the tool,
  \item evaluate website using framework and redevelop as needed and
  \item write up findings.
\end{enumerate}


\section{Contributions}

The key contributions to knowledge described in this thesis are:

\begin{itemize}
  \item Three pataphysical search algorithms (clinamen, syzygy and antinomy).
  \item A creative exploratory search tool demonstrating the algorithms in the form of a website \url{http://pata.physics.wtf}.
  \item A set of subjective parameters for defining creativity.
  \item An objective framework for evaluating creativity.
\end{itemize}

\begin{comment}
  abusing tech in creative ways can yield useful results
  pataphysics = Creativity

  combining the pseudo philosophy of pataphysics with semantically structured algorithms which use programming APIs and computational linguistics to produce original creative works.

  overturn expectations
  subvert browsing
  undermine relevance
  corrupt results
\end{comment}


\section{Publications}

\textbf{Fania Raczinski}, Dave Everitt (2016) \emph{`Creative Zombie Apocalypse: A Critique of Computer Creativity Evaluation'}. Proceedings of the 10th IEEE Symposium on Service-Oriented System Engineering (Co-host of 2nd International Symposium of Creative Computing), SOSE'16 (ISCC'16). Oxford, UK. Pages 270--276.

\textbf{Fania Raczinski}, Hongji Yang and Andrew Hugill (2013) \emph{`Creative Search Using Pataphysics'}. Proceedings of the 9th ACM Conference on Creativity and Cognition, CC'13. Sydney, Australia. Pages 274--280.

Andrew Hugill, Hongji Yang, \textbf{Fania Raczinski} and James Sawle (2013) \emph{`The pataphysics of creativity: developing a tool for creative search'}. Routledge: Digital Creativity, Volume 24, Issue 3. Pages 237--251.

James Sawle, \textbf{Fania Raczinski} and Hongji Yang (2011) \emph{`A Framework for Creativity in Search Results'}. The 3rd International Conference on Creative Content Technologies, CONTENT'11. Rome, Italy. Pages 54--57.

Please note that a full list of talks, exhibitions and publications is available in appendix~\ref{app:pub}\marginnote{§~\ref{app:pub}}.


\section{The Hitchhiker's Guide to this Thesis}

% \begin{framed}
%   % !TEX root = ../main.tex

\chapter{Hitchhiker's Guide}
\label{ch:guide}

\startcontents[chapters]

A leafy grove in the park, \\
and a Heart too wide to close in, \\
to show mercy towards, \\
des deux routes d'orient et d'Occident.

They should o'erthrow quite flat down dead th'empire, \\
par devant expert, \\
designates him as above the grade of the common sailor, \\
my uncle was by my side.

Ne point abolis par une structure posterieure el superieure de, \\
like eyes immense and feverish open wide, \\
which was pushing from the other side.

And several articles necessary in a balloon of extraordinary dimensions, \\
aurora streaks the sky with orient light, \\
it is my design to show you.


\vfill
\minicontents
\newpage

% 
% \begin{frame}
% \pointthis{There is}{Corot} a\pointthis{beautiful flower}%
% {Van Gogh}\pointthis{in}{Matisse}\pointthis{the garden}{Monet}.
% \end{frame}


\section{Chapter Poetry}

Each poem is generated using \url{pata.physics.wtf} with the chapter title as keyword.


\section{Margin Notes}

The different symbols used in margin notes are as follows.

\begin{description}
  \item [\faicon{table}] Represents a table.
  \item [\faicon{option-group}] Represents a figure.
  \item [§] Represents a chapter.
  \item [\faicon{picture-o}] Represents an image.
\end{description}


\section{Part Spirals}

Each new thesis part contains a spiral based on a poem generated by \url{pata.physics.wtf} using the part title as keyword.


\section{Thesis Language}

This thesis is written in \tex.

\stopcontents[chapters]

% \end{framed}

% \includepdf[pages=-, frame,scale=.8, pagecommand={\thispagestyle{plain}}]{guide.pdf}

This document is organised into 6 parts which form the main logical structure of the thesis and each part contains several chapters. There are margin notes pointing to relevant chapters, sections, tables, figures or images throughout.

\subsection*{Part Spirals}

Each new thesis part contains a word spiral based on a poem generated by \url{pata.physics.wtf} using the a part of the title as keyword. They represent the pataphysical (Archimedean) spiral.

\begin{enumerate}
  \item Preface --- \emph{pre}
  \item Hello World --- \emph{hello}
  \item Tools of the Trade --- \emph{trade}
  \item The Core: Techno-Logic --- \emph{core}
  \item The Core: Techno-Practice --- \emph{practice}
  \item Meta-Logicalysis --- \emph{meta}
  \item Happily Ever After --- \emph{after}
  \item Postface --- \emph{post}
\end{enumerate}


\subsection*{Chapter Poetry}

Each chapter opens with a poem generated by \url{pata.physics.wtf} using a part of the chapter title as keyword.

\begin{enumerate}
  \item Introduction --- \emph{intro}
  \item Inspirations --- \emph{inspiration}
  \item Methodology --- \emph{method}
  \item Pataphysics --- \emph{pata}
  \item Creativity --- \emph{creativity}
  \item Technology --- \emph{technology}
  \item Evaluation --- \emph{evaluation}
  \item Foundations --- \emph{foundation}
  \item Interpretation --- \emph{interpretation}
  \item Implementation --- \emph{implementation}
  \item Applications --- \emph{application}
  \item Patanalysis --- \emph{patanalysis}
  \item Aspirations --- \emph{aspirations}
  \item Observations --- \emph{observations}
\end{enumerate}

\todo{say more, check keywords, potentially generate new poems}


\subsection*{Margin Notes}

The different symbols used in margin notes are as follows.

\begin{description}
  \item [\faicon{table}] Represents a table.
  \item [\faicon{object-group}] Represents a figure.
  \item [§] Represents a chapter.
  \item [\faicon{picture-o}] Represents an image.
\end{description}

\todo{say more, add images to toc?}


\subsection*{Thesis Language}

This thesis is written in \LaTeX.
\todo{say more}


\spirals

\begin{description}
  \item[PREFACE] .
  \item[Part I] HELLO WORLD
  \item[Chapter 1] Introduction
  \item[Chapter 2] Inspirations
  \item[Chapter 3] Methodology
  \item[Part II] TOOLS OF THE TRADE
  \item[Chapter 4] Pataphysics
  \item[Chapter 5] Creativity
  \item[Chapter 6] Technology
  \item[Chapter 7] Evaluation
  \item[Part III] THE CORE: TECHNO-LOGIC
  \item[Chapter 7] Foundations
  \item[Chapter 8] Interpretation
  \item[Part IV] THE CORE: TECHNO-PRACTICE
  \item[Chapter 9] Implementation
  \item[Chapter 10] Applications
  \item[Part V] META-LOGICALYSIS
  \item[Chapter 11] Patanalysis
  \item[Chapter 12] Aspirations
  \item[Part VI] HAPPY END
  \item[Chapter 13] Observations
  \item[POSTFACE] .
\end{description}

\todo{update and describe each section briefly}

\stopcontents[chapters]
