% !TEX root = ../main.tex

\chapter{Introduction}
\label{ch:intro}

\startcontents[chapters]
\minicontents

% \emph{Part of this chapter has been described in a journal article in Digital Creativity in 2013, and I presented a paper at the Creativity and Cognition conference 2013 in Sydney.} \marginpar{§¶}

\brule % chktex 1

% {\color{red} HELLO WORLD}% COLOR RED

% \begin{tikzpicture}
% \draw [domain=0:25.1327,variable=\t,smooth,samples=55]
%     plot[mark=*,mark options={fill=white}] ({\t r}: {0.002*\t*\t});
% \end{tikzpicture}
%
% \begin{tikzpicture}[
%     decoration={
%     reverse path,
%     text along path,
%     text path start scale=1.5,
%     % text path start scale=3,
%     text path end scale=0,
%     text={feeling a movement of pity, discovered the induction coil, cette irraisonnee induction, and entered the opening in the wall. only by some recherche movement, apres coup et sous forme d'introduction, opening his seized manuscript, the enemy made within the enclosure of the vineyard. which he had thrown off at the beginning of his labor, in opening so exactly at the, than the thirst of my paternity. We can then start at once, and whose informing voice had consigned me to the hangman, as any person at all conversant with authorship may satisfy himself at.}}
% ]
% \draw [decorate]
%     (0.5\textwidth, 0) % horizontal. circle then line to side
%     % (0, 0.5\textwidth) % vertical. line from top down to circle
%     % \foreach \i [evaluate={\r=(\i/2000)^2}] in {0,5,...,2880}{ -- (\i:\r)};
%     \foreach \i [evaluate={\r=(\i/2500)^2}] in {0,10,...,3500}{ -- (\i:\r)};
% \useasboundingbox (-2.75,-2.75) rectangle (2.75,2.75);
% % \useasboundingbox (-5,-5) rectangle (5,5);
% \end{tikzpicture}

\brule % chktex 1

This thesis describes `Algorithmic Meta-Creativity'. More precisely it is about using creative computing to achieve computer creativity.

The project is transdisciplinary; it is heavily inspired by the absurd french pseudo-philosophy pataphysics and draws from a wide range of disciplines such as computer science, psychology, linguistics, literature, art and poetry, languages and mathematics.

The research involved includes exploring what it means to be creative as a human, how this translates to machines and how pataphysics relates to creativity.

The outcome is presented as a website -\url{pata.physics.wtf}- written in 5 different programming languages\footnote{Python, HTML, CSS, Jinja, JavaScript}, making calls to 6 external Web services\footnote{Microsoft Translate, WordNet, Bing Image Search, Getty, Flickr, YouTube}, in a total of over 3000 lines of code\footnote{2864 lines of code, 489 lines of comments - as of 08 Dec 2015}. It's main purpose is to demonstrate three creative `patalgorithms' in action in the context of exploratory information retrieval. A browsing rather than a search engine, it presents results in various formats such as sonnetts. Immediate inspirations come from fictional character `Doctor Faustroll' created by french absurdist Alfred Jarry, the `Celestial Emporium of Benevolent Knowledge' by magical realist Jorge Luis Borges and `A Hundred Thousand Billion Poems' by Pataphysician and Oulipian Raymond Queneau amongst others.

The system behind \url{pata.physics.wtf} partially automates the creative process (generating novel combinations, transforming conceptual spaces, etc).

Using pataphysics as a fuel for creativity through the medium of subjective pataphysical algorithms I fuse objective programming with creativity.


Current literature shows that creativity, specifically in regards to computers, is hard to measure objectively. I argue that the most appropriate way to approach this is by looking at five subjective constraints simultaneously (person, process, product, place, purpose) in a holistic manner.

\begin{fcom}
  automating creative process (randomising poems, writers go through drafts - i automate this process. poems = combinatorial creativity visualised.
  From calculated relevance to creative detour.
\end{fcom}

\begin{fcom}
  noise\\
  basins of attraction\\
  break symmetry\\
  creativity is not linear\\
  phase shift\\
  \autocite{Everitt2011}
\end{fcom}

\grule

original idea:
build pataphysical web search tool using semantic web tech.
create an ontology of creativity for Computers

why did i not use semantic web stuff?
semantic web is about agreed ontologies which oppose surprises. a creative ontology would work around that but needs the structure of the semantic web in place (rdf etc) which is hypothetical atm and not realistically implementable. SW is about standards. pataphysics is about breaking standards (exceptions etc).

bridge: how do current search engines work? they prioritise revelvance using pagerank algorithms etc. happens at crawling time. pataphysics isnt about relevance. (index is ranked)

pataphysics cant be ranked. need for neutrality in index but creative ways to retrieve matches for query.
but then changed to focus on the concept of searching/browsing (in itself, rather than part of a system architecture) and ranking as a creative process.
pataphysicalisation happens at query time between query and index. (index is neutral)

project was to build a prototype that proves these ideas.
my eventual approach was to take elemnts from both the ontology idea and the relevance ranking IR way and combine/redeploy them in a new way using pataphysics that would yield results designed to foster/inspire creativity.


\todo{finish introduction}


\section{Motivations}

Assumptions: computers can be creative.
\todo{complete assumptions, do I have any? or is the point to answer those as research questions?}

problem: can computers be considered creative? Can pataphysics facilitate creativity? Can computer creativity be objectively measured?

motivation: personal background? comp sci + arts and creativity
\todo{do I discuss personal interest in motivation?}

context: creative computing is about a creative process (with a potential creative output) whereas comp creativity is about a creative product using traditional means. comp creativity is on the rise such as AI is `emerging' still.

\grule

Why pataphysics?
Pataphysics is highly subjective and particular and is as such very suitable for this kind of transformation from relevant to creative.
Pataphysics can provide some useful techniques that are very suitable for creative computing.

\begin{quote}
  Purposive without purpose (Kant)

  ``Fourth, through aesthetic judgments, beautiful objects appear to be 'purposive without purpose' (sometimes translated as 'final without end'). An object's purpose is the concept according to which it was made (the concept of a vegetable soup in the mind of the cook, for example); an object is purposive if it appears to have such a purpose; if, in other words, it appears to have been made or designed. But it is part of the experience of beautiful objects, Kant argues, that they should affect us as if they had a purpose, although no particular purpose can be found.''\footnote{Kant, chapter 2a: \url{http://www.iep.utm.edu/kantaest/}}
\end{quote}


WHY NOT JUST RANDOMNESS?????
there has to be an injection of meaning at some point
random results are easy. but they might not be focused enough.

% It will challenge the binary logic that dominates the world of programmers and concentrate on what they would view as an illogical and unstructured system going against all standards.


\section{Questions}

closed
\begin{itemize}
  \item Can a creative process be automated or emulated by a computer?
  \item
  \item inspirational rather than informational (procedural)?
  \item How can we get search results that are unexpected and yet make sense?
  \item How can we present search results in a creative and pataphysical way?
  \item How does Pataphysics relate to creative computing?
  \item How can we use Pataphysics as inspiration for search ranking?
  \item How can we write a specifically creative algorithm?
\end{itemize}

open
\begin{itemize}
  \item How can we rank search results but still be true to Pataphysics philosophy?
  \item How can we represent and structure data to reflect its context, meaning and subjectivity?
  \item What does it mean for search results to be creative/relevant?
  \item Can computers be creative?
  \item What does it actually mean to be creative even for a human being, etc
  \item Is pataphysics creative?
  \item What is a relevant search result?
  \item Is creativity irrelevant?
\end{itemize}

\todo{answer research questions in conclusion}

\section{Process-ions}

This project combines research in science and art. It is an interdisciplinary research project.

computational linguistics (wordnet)
natural language processing (index, dameraulevenshtein)
creative computing
literature (pataphysics, oulipo)



This project has roots in disciplines such as Computer Science and Humanities.
\begin{description}
  \item [Information Retrieval]: Software Engineering, Semantic Web
  \item [Pataphysics]: Literature, Philosophy, Ontology
  \item [Creativity]: Cognitive Science, Artificial Intelligence
\end{description}

In regards to my project:
\begin{itemize}
  \item A concept implementation method is used with a descriptive-other approach
  \item A qualitative investigation into if and why the proposed search results are useful will be done
  \item Following experimental methodologies, to evaluate the proposed new solution to the problem of creative search
\end{itemize}

\begin{description}
  \item [Epistemology]: Subjective/Argumentative
  \item [Methodology]: Experimental, Interpretative, Qualitative
  \item [Methods]: Concept implementation, (Heuristic) Evaluation
\end{description}


\section{Product-ions}

\begin{itemize}
  \item Design a tool for creative searching on the Web
  \item Design pataphysics inspired algorithms to model creativity in this tool
  \item Produce a proof-of-concept prototype
  \item Propose a framework for evaluating and interpreting creative search results
\end{itemize}

\url{http://pata.physics.wtf} \marginpar{!!!}
\url{http://pata.fania.eu} \marginpar{!!!}


\section{Contributions}

abusing tech in creative ways can yield useful results
pataphysics = Creativity

combining the pseudo philosophy of pataphysics with sematically structured algorithms which use programming APIs and computational linguistics to produce original creative works.

patalgorithms

web presence

overturn expectations
subvert browsing
undermine relevance
corrupt results

5 Ps


\section{Publications}

James Sawle, \textbf{Fania Raczinski} and Hongji Yang (2011) \emph{``A Framework for Creativity in Search Results''}. The 3rd International Conference on Creative Content Technologies, CONTENT'11. Rome, Italy. Pages 54--57. \autocite{Sawle2011}

Andrew Hugill, Hongji Yang, \textbf{Fania Raczinski} and James Sawle (2013) \emph{``The pataphysics of creativity: developing a tool for creative search''}. Routledge: Digital Creativity, Volume 24, Issue 3. Pages 237--251. \autocite{Hugill2013d}

\textbf{Fania Raczinski}, Hongji Yang and Andrew Hugill (2013) \emph{``Creative Search Using Pataphysics''}. Proceedings of the 9th ACM Conference on Creativity and Cognition, CC'13. Sydney, Australia. Pages 274--280. \autocite{Raczinski2013}

Please note that a full list of talks, exhibitions and publications is available at appendix \ref{app:pub}.


\section{The Hitchhiker's Guide to this Thesis}

\begin{description}
  \item[PREFACE] .
  \item[Part I] IN THE BEGINNING\ldots
  \item[Chapter 1] Introduction
  \item[Chapter 2] Methodology
  \item[Part II] IN A GALAXY FAR FAR AWAY\ldots
  \item[Chapter 3] Pataphysics
  \item[Chapter 4] Creativity and Computers
  \item[Chapter 5] IR and NLP
  \item[Part III] THE CORE:\@ TECHNO-LOGIC
  \item[Chapter 6] Theoretical Foundations
  \item[Chapter 7] Practical Implementation
  \item[Chapter 8] Impact and Applications --- Case Study
  \item[Part IV] INTECHNOIL-LOGICALYSIS
  \item[Chapter 9] Interpretation / Evaluation
  \item[Chapter 10] Patacritical Analysis
  \item[Part V] HAPPY END
  \item[Chapter 11] Aspirations
  \item[Chapter 12] Observations
  \item[POSTFACE] .
\end{description}

\stopcontents[chapters]
