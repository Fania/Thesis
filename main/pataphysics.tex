% !TEX root = ../main.tex

\chapter[Pataphysics]{'Pataphysics}
\label{ch:pataphysics}

\emph{Part of this research has been described in a journal article in Digital Creativity in 2013, and I presented a paper at the Creativity and Cognition conference 2013 in Sydney.}

\grule

\begin{quote}
  "To understand 'pataphysics is to fail to understand 'pataphysics." \citep{Hugill2012a}
\end{quote}

It is probably impossible to define pataphysics in one sentence. There is no definition that does justice to what pataphysics really means and no single definition that is truer than any other. In fact, the college of pataphysics in France itself has published a book \citep{Brotchie2003} with over 100 definitions that they all call equally valid. This chapter therefore begins with a selection of definitions from that book (quoting their original sources).

\begin{quote}
  "Pataphysics … is the science of that which is superinduced upon metaphysics, whether within or beyond the latter's limitations, extending as far beyond metaphysics as the latter extends beyond physics. […] Pataphysics will be, above all, the science of the particular, despite the common opinion that the only science is that of the general. Pataphysics will examine the laws governing exceptions, and will explain the universe supplementary to this one. […] DEFINITION: Pataphysics is the science of imaginary solutions, which symbolically attributes the properties of objects, described by their virtuality, to their lineaments." (Alfred Jarry, "Exploits and Opinions of Dr Faustroll, Pataphysician" written in 1897-8 and published posthumously in 1911) \citep{Jarry1996}
\end{quote}

\begin{quote}
  "'Pataphysics is patient; 'Pataphysics is benign; 'Pataphysics envies nothing, is never distracted, never puffed up, it has neither aspirations nor seeks not its own, it is even-tempered, and thinks not evil; it mocks not iniquity: it is enraptured with scientific truth; it supports everything, believes everything, has faith in everything and upholds everything that is." ("Épanorthose sur le Clinamen moral", Cahiers du Collège de 'Pataphysique, 21, 22 Sable 83 (29 December 1955 vulg.)) \citep{Brotchie2003}
\end{quote}

\begin{quote}
  "'Pataphysics passes easily from one state of apparent definition to another. Thus it can present itself under the aspect of a gas, a liquid or a solid." (Patafluens 2001, Istituto Patafisico Vitellianese, Viadana, 2002) \citep{Brotchie2003}
\end{quote}

\begin{quote}
  "'Pataphysics, 'the science of the particular', does not, therefore, study the rules governing the general recurrence of a periodic incident (the expected case) so much as study the games governing the special occurrence of a sporadic accident (the excepted case). […] Jarry performs humorously on behalf of literature what Nietzsche performs seriously on behalf of philosophy. Both thinkers in effect attempt to dream up a 'gay science', whose joie de vivre thrives wherever the tyranny of truth has increased our esteem for the lie and wherever the tyranny of reason has increased our esteem for the mad." (Christian Bök, 'Pataphysics, The Poetics of an Imaginary Science, Northwestern University Press, 1ll., 2002) \citep{Bok2002}
\end{quote}

\begin{quote}
  \begin{multicols}{2}
  La pataphysique est la fin des fins.\\
  La pataphysique est la fin des faims.\\
  La pataphysique est la faim des fins.\\
  La pataphysique est le fin du fin.
  \par \vfill \columnbreak
  \begin{flushright}
  'Pataphysics is the end of ends.\\
  'Pataphysics is the end of hunger.\\
  'Pataphysics is the hunger for ends.\\
  'Pataphysics is the finest of the fine.
  \end{flushright}
  \end{multicols}
  (The first motto is that of the official Collège notepaper. Its three variants have appeared in elsewhere in Collège publications. - Collège de 'Pataphysique) \citep{Brotchie2003}
\end{quote}

\begin{quote}
  "The branch of philosophy that deals with an imaginary realm additional to metaphysics. " (Oxford Dictionary)
\end{quote}

\begin{quote}
  Epiphany – "to express the bursting forth or the revelation of pataphysics" Dr Sandomir \citep[p.174]{Hugill2012a}
\end{quote}

I divided my research about pataphysics into four approaches. The first: learn about its inventor - Alfred Jarry. The second: read his work. The third: read what others have to say about pataphysics. The fourth: read other literature that could be classed as pataphysical. Eventually I ended up seeing pataphysics in everything, there was no escaping it anymore. I had turned a Pataphysician.

Personally, when I try to explain pataphysics to laymen, I use the last scene of the movie "Men In Black" as an example. The scene zooms out further and further, from a close up of Will Smith, to his car, to a shot of the city from above, to a shot of the earth, the galaxy, the universe and finally it is revealed that the universe is in a marble that is being toyed with by an alien. This is a good example of different layers of abstraction – Will Smith in his car represents the physical layer, the universe the metaphysical layer and the alien marble the pataphysical layer. The outro scene can be seen on YouTube \footnote{\url{http://youtu.be/1QPll-TKaEE}}.

\begin{figure}[htb] % (here, top, bottom, page)
  \centering
  \includegraphics[width=\linewidth]{images/mib}
  \caption[Men in Black]{Men in Black Screenshots of Ending Sequence}
  \label{fig:MIB}
\end{figure}

\section{Consciousness}

\begin{quote}
  Jarry was "attempting to transcend his own existence." \citep{Hugill2012}
\end{quote}

\begin{quote}
  "It is certainly true that making life 'as beautiful as literature' was one of his goals." \citep{Hugill2012}
\end{quote}

Alfred Jarry was born in Laval, Mayenne, France in 1873 and died in Paris in 1907, at the age of 34. He was known as a poet, dramatist, novelist and journalist but also as a graphic artist. His hobbies included entomology, fishing, cycling, fencing, shooting and drinking.

\begin{figure}[htb]
  \centering
  \begin{minipage}{.275\linewidth}
    \includegraphics[width=\linewidth]{jarry06}
  \end{minipage}
  \hspace{.05\linewidth}
  \begin{minipage}{.275\linewidth}
    \includegraphics[width=\linewidth]{jarry01}
  \end{minipage}
  \hspace{.05\linewidth}
  \begin{minipage}{.275\linewidth}
    \includegraphics[width=\linewidth]{jarry05}
  \end{minipage}
  \caption[figures1-3]{FelixVallotton + jarry + jpicasso}
  \label{img123}
\end{figure}

He went to school in Rennes, where his physics teachers Félix-Frédéric Hébert left such a big impression on Jarry that he would later be his inspiration for Père Ubu. He passed his baccalauréat with 17 and moved to Paris to attend the lycée Henri IV in preparation to apply for admission to the École Normale Supérieure but eventually gave upon the entrance exam after several unsuccessful attempts. He met another teacher at the lycée, this time a philosophy teacher called Henri Bergson, who inspired him greatly. He published his first collection of poems in 1893, aged 20, the year his mother died. One of his classmates there described him as follows.

\begin{quote}
  "[…] I found Jarry's mental processes disturbing. When he let himself go he seemed in thrall to a torrent of words outside his control. It was no longer a person speaking, but a machine controlled by a demon. His staccato voice, metallic and nasal, his abrupt puppet-like gestures, his fixed expression and uncontrolled flood of language, his grotesque and brilliant turns of phrases, ended up provoking a feeling of disquiet. He was informed, intelligent, and discriminating; he was good person, secretly kind, perhaps even shy beneath it all […] but his originality resembled nothing short of a mental anomaly." (Jarry's classmate at the lycée Henri IV: Gandilhon Gens-d'Armes "Alfred Jarry au lycée Henri IV" Les Marges, XXIII, 91 (15 Jan 1922) as quoted in \citep{Brotchie2011})
\end{quote}

He was at the centre of the avant-garde movement in Paris around that time, at the centre of the Tuesday meetings of the Mercure de France (a literary magazine run by Alfred Valette and his wife Rachilde, who soon became a sort of substitute family to Jarry who was roughly 15 years younger than them). Being rather misogynist at times and homosexually inclined, Rachilde was one of his very few female friends.

The following year, 1895, he briefly joined the army in the 101st Infantry, after having dodged it by being an enrolled student at the lycée. He followed rules there pedantically but hated the loss of his individualism. According to Brotchie, he "chose subservience, but subservience taken to the point of parody: the pataphysical solution to the problem of obedience"\citep{Brotchie2011}. Probably the only thing he enjoyed there was the fencing and shooting training. He looked funny in the uniform that was too big for him being so small (5'3'') so he was eventually excused from parades and after a few months he was allowed to leave to Paris frequently. He was discharged in December 1895 on medical grounds: gallstones. It is not unlikely that he faked the illness by drinking picric acid.

\begin{figure}[htb] % (here, top, bottom, page)
  \centering
  \includegraphics[height=0.3\textheight]{images/ubu}
  \caption[Ubu]{Woodprint of Ubu by Alfred Jarry}
  \label{fig:UBU}
\end{figure}

His father had died just two months earlier and had left him a small inheritance, which he spent mostly on publishing his very own magazine dedicated to symbolist wood carvings, the Perhinderion. He had previously co-edited the magazine L'Ymagier with Remy de Gourmont between 1893 and 1894. He joined Aurélien Lugné-Poë as his secretary (his only ever real job) at the Théâtre de l'Œuvre after his discharge at the army, where he would pour his utmost attention to putting his Ubu play on the stage. He also played a small role in the production of Peer Gynt at the Œuvre earlier in 1896. The printed version of Ubu Roi appeared in Le Livre d'Art in the middle of the year with Jarry's carved woodcut image of Ubu that became so popular. The première took place on 10th December that year and caused an outrage in the audience after the first word: "merdre" (sometimes translated as "pshit"). Jarry had previously arranged for certain friends to counter any reaction of the general audience and to prevent under all circumstances for the play to reach its conclusion. The performance went according to plan. The uproar after the first word was uttered was immense, the performance had to be interrupted at times to calm the audience and it finished in shouts of praise, protest and insults. There were no further performances but the event was considered historic even at the time and is now widely seen as the first "modern" play \citep[p.168-169]{Brotchie2011}. And as Dave Walsh puts it: "Movements such as Dadaism, Surrealism, Futurism, Expressionism Cubism, Theatre of the Absurd - all owe debts to [Jarry's] works." \citep{Walsh2001}

Although Ubu's mannerism of speech was originally imitating Jarry's, as suggested by Lugné-Poë \citep[p.155]{Brotchie2011}, Jarry continued to adapt Ubu's mannerisms.

\begin{quote}
  "Those who knew him said that his nauseating appearance hid a youth who was stubborn yet shy, proud and little full of himself, but good-natured and ingenuous behind his cynicism, one who was fiercely independent and rigorously honest." (Henri de Régnier, as quoted in \citep[p.181]{Brotchie2011})
\end{quote}

\begin{quote}
  "Alfred Jarry had a very particular way of speaking to that was disconcerting to those who heard it for the first time. He said 'we', when referring to himself, and substituted verbs for nouns, in imitation of ancient Greek. Example: 'celui qui soufflé' (that which blows) for the wind, and 'celui qui se traîne' (that which crawls along) for the train, even if it was an express! This made conversation somewhat complicated, not least because of the rapidity of his delivery." (Rachilde, as quoted in \citep[p.181]{Brotchie2011})
\end{quote}

\begin{quote}
  "Alfred Jarry was a man of letters to an unprecedented extent. His smallest actions, his childish pranks, everything he did was literature. His whole life was shaped by literature, and only by literature." (Apollinaire, as quoted in \citep[p.307]{Brotchie2011})
\end{quote}

Jarry spent the next few years writing. He had spent all his inheritance on the publication of his magazine and the production of Ubu Roi. It is during this time that he moved to his infamous tiny flat on the second-and-a-half floor. Jarry could just about stand upright but any guests had to crouch. He had no electricity or gas and no means of cooking \citep[p.195]{Brotchie2011}. In December 1897 he formed a marionette theatre with his friend Claude Terasse: the Théâtre de Pantins and they performed Ubu Roi in January 1898 without riots in the audience.

Jarry then gradually withdrew from the literary circles in Paris and spent more time in a little shack on the banks of the Seine near the village of Le Coudray. He started writing a regular review column for the Revue Blanche in 1900, the income of which he certainly needed much. There was a brief revival of the Ubu marionette play in the Cabaret des Quat'z'Arts in 1901.

Around 1904 he began drinking ether, the absinthe not strong enough anymore. In the winter of 1905 he was very ill, the cold and poverty not helping. In 1906, his friends became more and more concerned about his deteriorating health and eventually Valette and Saltas sent him to his sister Charlotte. He then spent some time in Paris and some in Laval at his sister's place over the next year. Jarry then died in November of 1907 of meningeal tuberculosis. His last request was for a toothpick.

\begin{quote}
  "He believes that the decomposing brain goes on working after death and it is its dreams that are Paradise." (Jarry 1906 in a letter to Rachilde \citep{Brotchie2007} - 'he' refers to himself)
\end{quote}

Studying Jarry's life gives certain insights into the man who created pataphysics and why he might have done so. Alastair Brotchie has written the probably most concise and recent biography of Alfred Jarry in English language and most of the information summarised here comes from this book \citep{Brotchie2011}. Roger Shattuck gives a very nice introduction about Jarry in relation to the time and place he lived in, in his book "The Banquet Years" \citep{Shattuck1959}. However, he does not focus on Jarry alone but rather on the time period and four personalities (Alfred Jarry, Henri Rousseau, Erik Satie and Guillaume Apollinaire) he chose as representatives of the era.


\subsection*{His Writing}

Jarry has written a good amount of texts in his short life and he didn't confine himself to a single category either. He wrote poems, novels, short stories, essays, art reviews, theatre reviews and plays and also produced translations of a few texts into French. Many of his texts were completely fictional, some had autobiographical aspects and some scientific and most of them had a sarcastic sense of humour. Trying to summarise Jarry's style of writing or attempting to interpret his whole body of work seems impossible though.

\begin{quote}
  "Jarry was an acknowledged classical scholar, had already worked as a reviewer of art and drama, had edited two art magazines, was up to date with modern scientific theory, especially physics, read widely in mathematics and psychology, and had an extensive basic knowledge of philosophy." \citep{Brotchie2011}
\end{quote}

James A Cutshall says that "instead of Jarry the man and the meaning of his literary endeavours becoming clearer with the passage of time, both have become increasingly indistinct" \citep[p.246]{Cutshall1988}. The intention of his thesis was to show the seriousness implied behind the humour in many of Jarry's novels, in order to give the author the merit he deserved. Cutshall wrote about Jarry's novels rather than simply seeing him as the playwright of the Ubu plays. He surveyed existing criticism about Jarry's texts and provided his own view on them. He immortalised Jarry by saying "whether or not this is the sort of 'éternité' sought by the heroes of Jarry's novels, it is certainly that which their author somewhat belatedly has found" \citep[p.248]{Cutshall1988}.

\begin{figure}[htb]
  \centering
  \includegraphics[height=0.3\textheight]{images/faustroll}
  \caption[Faustroll]{Faustroll illustration by Steve Morrison}
  \label{fig:FAUSTROLL}
\end{figure}\footnote{\url{http://bit.ly/1Q0VZW9}} % http://stevemorrisonillustration.blogspot.co.uk/2009/03/dr-faustroll.html

Cutshall was not the only one who has written about certain less-known texts by Jarry. Marieke Dubbelboer's thesis "Ubusing Culture" is also interesting in this regard since it concentrates completely on the "Almanachs du Père Ubu" (published in 1898 and 1901) \citep{Dubbelboer2009}. She was looking for keys to Jarry's poetics in those texts, which she says "seemed to defy labelling or literary norms" \citep[p.10]{Dubbelboer2009}. She claims the Almanacs to be quite radical and exemplary of his innovative poetics moving away from Symbolism and towards the Avant-Garde. In general she says his work "can be characterized as playful, elusive, paradoxical and provocative" \citep[p.197]{Dubbelboer2009} and his two Almanacs are the essence of his non-conformist attitude. They were written at a time of change for Jarry, when he withdrew from his usual circles in Paris and he published in new magazines, which links his change in writing according to Dubbelboer.

A list of his works can be found in the appendix \ref{app:jarry}.


\section{Intellection}

We will need to understand the essence of pataphysics to understand how it relates to creative computing.

Jarry first defined pataphysics in his book "Exploits and Opinions of Dr Faustroll, Pataphysician" written in 1898 and published posthumously in 1911 \citep{Jarry1996}. But the concept appeared as early as in 1893 in his prose text Guignol that won him a prize in the newspaper L'Echo de Paris and it appears in many of his writings. He originally intended to write a whole book called "Elements of Pataphysics" but only part of this appeared in Faustroll.

Zoe Corbyn gives a very simple short introduction for beginners of the topic in an article in the Guardian \citep{Corbyn2005} in 2005. She describes it like this:

\begin{quote}
  "Correct definitions are equivalent to wrong ones; all religions are on a par as imaginary and equally important; chalk really is cheese. It's an escape from reality - reminding us of just how idiotic the rules that dog our everyday existence are." \citep{Jarry1996}
\end{quote}

Jean Baudrillard has a few other definitions for pataphysics in his text\citep{Baudrillard2007}. According to him pataphysics is "the highest temptation of the spirit, the nail in the tire, the philosophy of the gaseous state, the science or the unique imaginary solution to the absence of problems" to name just a few.

Another rather strange interpretation of pataphysics is Asger Jorn's. He calls pataphysics a religion in the making \citep{Jorn1961}. He claims that since "natural religion is the spiritual confirmation of material existence", "metaphysical religion represents the establishment of an ever deepening rift between material and spiritual life." He refers to the idea of equivalence in pataphysics and the absolute and links them to religion. He says "the great merit of Pataphysics is to have confirmed that there is no metaphysical justification for forcing everybody to believe in the same absurdity".

Cruickshank \citep{Cruickshank} wrote a rather funny article on anti-matter. He links the creation of anti-matter atoms at CERN around 1996 with Jarry, saying that he had "beaten them to the punch" with his pataphysics.

Christian Bök \citep{Bok2002} tries to draw science and poetry together using pataphysics as the string that binds them. He compares Jarry and Nietzsche, saying Jarry performs humorously on behalf of literature what Nietzsche performs seriously on behalf of philosophy; both try to create an antiphilosophy \citep[p.9]{Bok2002}. He also claims that science and poetry have a similar history, undergoing the same four phases of distinct change but also that they have not evolved in sync with each other \citep[p.15]{Bok2002}.

\begin{description}
  \item [Animalistic phase]: signs exist long before being known, they are written by nature
  \item	[Mechanismic phase]: signs exist by being known, they are written by culture
  \item [Organismic phase]: signs evolve by being known, they are written across events by culture
  \item	[Cyborganismic phase]: signs evolve beyond being known, they are written as events by culture
\end{description}

\begin{quote}
  "Pataphysics is a surrational perspective that has had an extensive, yet forgotten, influence upon the canonic history of radical poetics. […] Not only does this avant-garde pseudoscience valorise whatever is exceptional and paralogical; it also sets the parameters for the contemporary relationships between science and poetry." \citep[p.27]{Bok2002}
\end{quote}

Bök also compares Jarry and Nietzsche in regards to perspectivism \citep[p.31]{Bok2002}. For Nietzsche reality is the effect of a dream world in which "there are many kinds of truths, and consequently there is no truth". And similarly for Jarry, reality is an aspect of ethernity in which "there are only hallucinations, or perceptions," and every "perception is a hallucination which is true". Both argue that no view is absolute as well and pataphysics argues that every viewpoint is dissolute, including its own because no view can offer a norm. Even Jarry's ethernity is nowhere and somewhere at the same time.

In Faustroll, Bök says, "Jarry parodies the discourse of such scientific luminaries, who attempt to demonstrate the utility of science through the dramaturgic performance of a mechanical experiment" \citep[p.29]{Bok2002}.

According to the Collège de 'Pataphysique, it is convention to use the apostrophe at the beginning of the word ('Pataphysics) only in reference to Jarry's texts, to the science of imaginary solutions as such. Used as an adjective or in a more unconscious way it is written without the apostrophe. Jarry himself just indicated that the word is preceded by the apostrophe to avoid a pun.

\begin{itemize}
  \item Vian, B. (2006). 'Pataphysics? What's That? (S. Chapman, Trans.). London: Atlas Press.\citep{Vian2006}
  \item Daumal, R. (2012). Pataphysical Essays. (T. Vosteen, Trans.). Cambridge, Massachusetts: Wakefield Press.\citep{Daumal2012}
  \item Brotchie, A. (Ed.). (1995). A True History of the College of  'Pataphysics - 1. (P. Edwards, Trans.). London: Atlas Press.\citep{Brotchie1995}
\end{itemize}


\subsection{Symbology}

Probably the most famous symbol of pataphysics is the grand gidouille, the big spiral on Ubu's fat belly. Not simply because it is a feature of Jarry's most popular creation but also because it represents one of the concepts of pataphysics itself: the antimony. The spiral can be interpreted as two spirals in one, the outer and the inner spiral. They represent the duality of pataphysics, the mutually incompatible in perfect harmony. The Collège de 'Pataphysique has adopted the spiral for its membership badges, in various colours and sizes for the different ranks of the college.

Another symbol of pataphysics is the green candle which refers to one of Jarry's last endeavours, published posthumously, a vast collection of his journalistic essays \citep{Hugill2012}. Some animals also symbolise pataphysics. The crocodile, the current vice-curator of the college is a crocodile named Lutembi \citep{Hugill2012}. Owls are another symbol; Jarry kept stuffed and live owls \citep[p.46]{Brotchie2011}[13 p46] in his flat. The chameleon is another, having the ability to change colour and looking in two directions at the same time.

\begin{figure}[htb]
  \centering
  \begin{minipage}{.275\linewidth}
    \includegraphics[width=\linewidth]{lutembi}
  \caption[Crocodile]{Crocodile from the CoP website}
  \label{img1}
  \end{minipage}
  \hspace{.05\linewidth}
  \begin{minipage}{.275\linewidth}
    \includegraphics[width=\linewidth]{spiral01}
  \caption[Gidouille]{The Grand Gidouille}
  \label{img2}
  \end{minipage}
  \hspace{.05\linewidth}
  \begin{minipage}{.275\linewidth}
    \includegraphics[width=\linewidth]{greencandle}
  \caption[Green Candle]{The green candle}
  \label{img3}
  \end{minipage}
\end{figure}

\begin{draft}
  We argue that pataphysics can facilitate creative computing. A pataphysical grammar consists of exceptions, syzygies, anomalies, clinamen, antinomies, contradictions, equivalents and imaginaries. Such constraints can transform the ways in which we may navigate and transform our conceptual space. Pataphysical concepts are likely to cause surprise and could therefore be considered unconventional and provocative.
\end{draft}


\subsection{Antimony}

The antimony is the mutually incompatible. It appears everywhere in Jarry's writings. It represents the duality of things, the echo or symmetry, the good and the evil at the same time. Examples are the plus minus, the faust-troll, the haldern-ablou, the yes-but, the ha-ha and the paradox.

The "Ha Ha", the only words Bosse-da-Nage ever utters in Faustroll, "is the idea of duality, of echo, of distance, of symmetry, of greatness and duration, of the two principles of good and evil." \citep{Hugill2012} Referring to the yes-but statement Hugill says "this may be taken as a standard pataphysical response to any proposition (including this one)." And most obviously the antimony can be seen in all the contradictions that pataphysics is so fond of.

The antinomy, in a pataphysical sense, is the mutually incompatible or paradox. Mutually contradictory opposites can and do co-exist in the pataphysical universe.


\subsection{Anomaly}

The anomaly is the exception. And exceptions are important in pataphysics. But then again everything is equal, so in a pataphysical world no exceptions would exist at all, or rather, everything would be equally exceptional. The anomaly disrupts and surprises. Hugill mentioned a great example of a collection of anomalies: the sourcebook project by William Corliss, who collects scientific papers that are anomalous. Bök says it is "the repressed part of a rule which ensure that the rule does not work" \citep[p.38]{Bok2002}.


\subsection{Sysygy}

The syzygy surprises and confuses. It originally comes from astronomy and denotes the alignment of three celestial bodies in a straight line. In a pataphysical context it is the pun. It usually describes a conjunction of things, something unexpected and surprising. Serendipity is a simple chance encounter but the syzygy has a more scientific purpose. Bök mentions Jarry saying that the fall of a body towards a centre is the same as the ascension of a vacuum towards a periphery \citep[p.42]{Bok2002}.

A syzygy both surprises and confuses. The concept originally comes from the field of astronomy where it denotes the alignment of three celestial bodies. In a pataphysical context it usually describes a conjunction of things, something unexpected and surprising. Unlike serendipity, a simple chance encounter, the syzygy has a more scientific purpose. A typical instance is the pun, which Jarry called the syzygy of words \citep{Jarry1996}. Next to being intentionally funny, puns demonstrate a clever use (or abuse) of grammar, syntax, pronunciation and/or semantics, often taken to a quite scientific level, such that without understanding of what is said and what is the intended meaning, the humour of the pun might be lost.


\subsection{Clinamen}

The clinamen is the unpredictable swerve that Bök calls "the smallest possible aberration that can make the greatest possible difference"  \citep[p.43]{Bok2002}. He links it to Lucretius idea of an atom serving in its streamlined flow to create matter and to Epicurus' parenklisis. But he also points out similarities to ideas like the Situationists' "détournement", the reuse of pre-existing aesthetic elements and Hugill links it to the Dadaists' ready-mades and Oulipo's verbal games. An obvious example is Jarry's "merdre", a swerve of the French word for shit (merde).

The concept of the clinamen can be understood as an unpredictable swerve which Bök called the smallest possible aberration that can make the greatest possible difference \citep{Bok2002}. One of the most famous examples of a clinamen is Jarry's merdre (the first word in his Ubu plays). He squeezed an extra 'r' into the French word merde (meaning shit) and translates into something like pshit.


\subsection{Absolute}

The absolute is a reference to a transcended reality. Jarry talks about "ethernity" in Faustroll \citep[p.104]{Jarry1996}.


\subsection*{Others}

Other concepts that are pataphysical or can be linked to it in a sense are alchemy and quantum mechanics. Alchemy because of its laws or equivalence and the union of opposites \citep{Hugill2012} and quantum mechanics because of principles of uncertainty, indeterminacy and the idea of the multiverse of course.

\begin{comment}
  Because string theory is speculation based on ideas that are themselves speculative (i.e., theories ofgeneral relativity and quantum mechanics), string theory is not in fact physics, but 'pataphysics.
\end{comment}

\begin{comment}
  Likewise, string theory and quantum calculations are, increasingly, not descriptive of an actual reality, but are simply mathematical pataphors. (P. Lopez) \footnote{\url{http://www.urbandictionary.com/define.php?term=pataphysics}}
\end{comment}


\section{Unconsciousness}


\subsection{Oulipo}

\todo{Finish section here. references and all}

\begin{quote}
  Potential literature is "the search for new forms and structures that may be used by writers in any way they see fit." Raymond Queneau (p2)
\end{quote}

\begin{quote}
  "The Oulipo's goal is to discover new structures and to furnish for each structure a small number of examples." François Le Lionnais (p3)
\end{quote}

\begin{quote}
  "a formal quest" Warren Motte (p3)
\end{quote}

\begin{quote}
  "Erecting the aesthetic of formal constraint, then, the Oulipo simultaneously devalues inspiration." (p10)
\end{quote}

\begin{quote}
  "Three levels in the hierarchy of constraints:\\1 Minimal level: constraints on the language in which the text is written\\2 Intermedaite level: constraints on genre and certain literary norms\\3 Maximum level: consciously preelaborated and voluntarily imposed systems of artifice\\Oulipo is in the maximum level." François Le Lionnais (p11)
\end{quote}

\begin{quote}
  "Oulipian systems of formal constraint are often based on the alphabet." François Le Lionnais (p13)
\end{quote}

\begin{quote}
  "The nature of Oulipoan constraint is mathematical." François Le Lionnais
\end{quote}

\begin{quote}
  "The Oulipo is anti-chance" Claude Berge (p17)
\end{quote}

\begin{quote}
  "What is the objective of our work? To propose new 'structures' to writers, mathematical in nature, or to invent new artificial or mechanical procedures that will contribute to literary activity: props for inspiration as it were, or rather, in a way, aids for creativity." Raymond Queneau (p51)
\end{quote}

\begin{quote}
  "Aleatoricism is the incorporation of chance into the process of creation, especially the creation of art or media. The word derives from the Latin word alea, the rolling of dice. It should not be confused with either improvisation or indeterminacy.[1]" Wikipedia
\end{quote}


\subsection{Borges}

You could argue that by reading other literature that is pataphysical (whether or not it was intended) one can learn something about pataphysics. Reading Borges \citep{Borges1964, Borges2004, Borges1957, Borges2010a, Borges2010b, Borges2000} is a good example. His text "The analytical language of John Wilkins" \citep{Borges2000} contains a brilliant example of pataphysical thinking and coincidentally a good example of the kinds of search results my search tool should hopefully produce.

Referring to a certain Chinese dictionary entitled "The Celestial Emporium of Benevolent Knowledge" he claims that animals can be divided into:

\begin{enumerate}
  \item	those belonging to the Emperor
  \item	those that are embalmed
  \item	those that are tame
  \item	pigs
  \item	sirens
  \item	imaginary animals
  \item	wild dogs
  \item	those included in this classification
  \item	those that are crazy-acting
  \item	those that are uncountable
  \item	those painted with the finest brush made of camel hair
  \item	miscellaneous
  \item	those which have just broken a vase
  \item	those which, from a distance, look like flies
\end{enumerate}

This kind of categorisation has also been discussed by Foucault in his book "The Order of Things" \citep{Foucault1966}.

% \section*{Summary}
%
% Alfred Jarry's way of life has certainly contributed to the phenomenon of Ubu and pataphysics. And his life and experiences and influences have formed pataphysics. His teachers, his schoolmates and friends, his literary circles in Paris, his military service, his staging of Ubu, his alcoholism, and his illness have all contributed to his character, his work and his pataphysics.
% pataphysics is a pseudo-science and pseudo-philosophy. Concepts of pataphysics are syzygy, anomaly, antimony, absolute and clinamen. As such pataphysics provides a foundation for becoming the rules for the proposed search tool.
