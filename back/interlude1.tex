% !TEX root = ../main.tex

\pagestyle{empty}

\chapter*{INTERLUDE I}
\label{interlude1}

% introduction
% inspiration
% methodology
% pataphysics
% creativity
% technology
% evaluation


\begin{quotation}
    Chance encounters are fine, but if they have no sense of purpose, they rapidly lose relevance and effectiveness. The key is to retain the element of surprise while at the same time avoiding a succession of complete non-sequiturs and irrelevant content \sourceatright{\autocite{Hendler2011}}
\end{quotation}

\begin{quotation}
    Conducting scientific research means remaining open to surprise and being prepared to invent a new logic to explain experimental results that fall outside current theory. \sourceatright{\autocite{Jarry2006}}
\end{quotation}

\begin{quotation}
    Only those who attempt the absurd achieve the impossible. \sourceatright{(attributed to M.C. Escher)}
\end{quotation}

\begin{quotation}
    A great truth is a truth whose opposite is also a great truth. Thomas Mann \sourceatright{\autocite[as cited in][]{Wickson2006}}
\end{quotation}

\begin{quotation}
    Heisenberg's Uncertainty Principle is merely an application, a demonstration of the Clinamen, subjective viewpoint and anthropocentrism all rolled into one. \sourceatright{\autocite{Jarry2006}}
\end{quotation}

\begin{quotation}
  Epiphany – `to express the bursting forth or the revelation of pataphysics' \sourceatright{Dr Sandomir \autocite[p.174]{Hugill2012a}}
\end{quotation}

\clearpage