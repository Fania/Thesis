% !TEX root = ../main.tex

\pagestyle{empty}

\chapter*{INTERLUDE I}
\label{interlude1}

% introduction
% inspiration
% methodology
% pataphysics
% creativity
% technology
% evaluation

\begin{quotation}
  [\ldots] through aesthetic judgments, beautiful objects appear to be ``purposive without purpose'' (sometimes translated as ``final without end''). An object's purpose is the concept according to which it was made (the concept of a vegetable soup in the mind of the cook, for example); an object is purposive if it appears to have such a purpose; if, in other words, it appears to have been made or designed. But it is part of the experience of beautiful objects, Kant argues, that they should affect us as if they had a purpose, although no particular purpose can be found. \sourceatright{\autocite[ch.2a]{Burnham2015}}
\end{quotation}

\begin{quotation}
    Chance encounters are fine, but if they have no sense of purpose, they rapidly lose relevance and effectiveness. The key is to retain the element of surprise while at the same time avoiding a succession of complete non-sequiturs and irrelevant content \sourceatright{\autocite{Hendler2011}}
\end{quotation}

\begin{quotation}
    Conducting scientific research means remaining open to surprise and being prepared to invent a new logic to explain experimental results that fall outside current theory. \sourceatright{\autocite{Jarry2006}}
\end{quotation}



\clearpage