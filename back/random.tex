% !TEX root = ../main.tex

\chapter{Random}
\label{app:random}


\section{Random Sentences}
\label{s:appsentences}

\begin{minted}{haskell}
in posthumous collaboration
the decomposing brain goes on working after death and it is its dreams that are Paradise
plagiarism by anticipation
the applause of silence is the only kind that counts
to understand pataphysics is to fail to understand pataphysics
duration is the transformation of a succession into a reversion
god is the tangential point between zero and infinity
laughter is born out of the discovery of the contradictory
ha ha
the aesthetic of formal constraint
the unique imaginary solution to the absence of problems
the contemporary relationship between science and poetry
a huge and elaborately constructed hoax
only those who attempt the absurd achieve the impossible
the random is opposed to the deterministic
pure multiplicity irreducible to any other sort of unity
persistence and perseverance to buttress a fleeting existence
enfolding a subject laterally, associatively
a doctrine of correspondences, counterpoised by the exotic charm of another system of thought
double negative is necessary to stop the mind believing
the absence of contradictory evidence is not proof of a theory's validity
very wrong in very important ways
no one point of view is final
unification of opposites
an athletic aesthetics of intuitive and instantaneous judgements
constantly diverted from any objective by the very progress which their energy sustains
imagination envisions the reconciliation of the individual with the whole
behind the illusion lies knowledge
a biomolecular bibliomecha of breathtaking beauty
indubitably coherent yet absolutely nonsensical
stylistic and formal experimentation can not be dismissed as purely apolitical
variable phoneme sequences in suspension within a cloud of relative epi-cultural etymologies
speculative solutions for imaginary problems
nonsense is nonsense only when we have not yet found the point of view from which it makes sense
laughter is the discord between tensions being resolved
read with intention to rewrite
a fractal geometry of momentums
minimizing energy, crystallizing latent structure, pleasure is understood as a practice, and accumulates as experience
don't fall out of love
the `something like', the pseudo
the text transforms itself as soon as it is understood
adaptation of archaic mental structures to new environments
a beautifully controlled yet hideously wasteful catastrophe
driven by compulsive urgency to constantly reconceive the whole idea
writing the unrightable wrongs
prisoners of conscience
machina sapiens negotiating the transformation of what is mortal into what is immortal
database hyper-archive applications stimulating relaxation
smelling of the rain it falls on the way down
a thousand ways to greet the dawn
how far up the chain can you put this without ambiguity
a gentle kitten is licking the inside of my heart
was this constrained by you, or restrained by the concept
adjoining always antiquated permutation
joint ventures can go too far away
nowhere to be found here and elsewhere
engender links to a balancing veneration
coerce to do as a sizeable unprocessed primer
plain up be a best concoction in the words
the farcical pandemonium of technology
it is not true that there were any nails
this discovery opens the door into a completely new anti-world
extending as far beyond metaphysics as the latter extends beyond physics
turn the world upside down and inside out
the law of the ascension of a vacuum toward a periphery
the anti-world God not only plays dice, he spells his name backwards
in the absence of a butler, where does the gun fit in
space is defined by simultaneity
time is a flowing stream, a liquid in uniform rectilinear motion
space is a solid, a rigid system of phenomena
the deceleration of our habitual duration conserved by inertia
a perfect elastic solid
movement into the past consists in the perception of the reversibility of phenomena
relativity is absolute
all observations depend on viewpoint and the scale of the scientist
the clinamen, subjective viewpoint and anthopocentrism all rolled into one
the identity of opposites
making negatives do the work of positives
in this year eighteen hundred and ninety-eight
the twenty-seven equivalents
the virgin, the bright, and the beautiful today
the fifth letter of the first word of the first act
voices asymptotic towards death
an epiphenomenon is that which is superinduced upon a phenomenon
concerning the amorphous isle
like soft coral, amoeboid and protoplasmic
searching desperately under the quinuncial trees for the venerable absent one
the night computed its hours
a remarkable epizootic disease
the eternal nothingness
love looks exactly like an iridescent veil and assumes the masked face of a chrysalis
in a telepathic letter
homo est deus
$\infty$ -0 -a + a + 0 = $\infty$
with the aim of computing the qualities of the French
the inferno of subjectivity
\end{minted}


\section{Heisenberg Quote}
\label{s:heisenberg}

\begin{quotation}
  The overly forceful insistence on the difference between scientific and artistic cognition quite likely derives from the incorrect notion that concepts are firmly attached to `real objects', as if words had a completely clear and definite meaning in their relationship to reality and as if an accurate sentence, constructed from those words, could deliver an intended `objective' factual situation to a more or less absolute degree. But we know, after all, that language too only grasps and shapes reality by turning it into ideas, by idealizing it. Language, too, approaches reality with specific mental forms about which we do not know right away which part of reality they can comprehend and shape. The question about `right' or `wrong' may indeed be rigorously posed and settled within an idealization, but not in relation to reality. That is why the last measure available for scientific knowledge as well is only the degree to which that knowledge is able to illuminate reality or, better, how that illumination allows us `to find our way' better. And who could question that the spiritual content of a work of art too illumines reality for us and makes it translucent? One must come to terms with the fact that only through the process of cognition itself can we determine what we are to understand by `cognition'. That is why any genuine philosophy, too, stands on the threshold between science and poetry. \sourceatright{\autocite{Heisenberg1942}}
\end{quotation}


\section{Digital Humanities Methodology Field Map}
\label{s:dhmap}

The full \textit{Field map of digital humanities: emerging methods and genres} by Burdick et al. \citeyear{Burdick2012}.

\begin{multicols}{2}\raggedright
\begin{itemize}
  \item enhanced critical curation
  \begin{itemize}
    \item digital collections
    \item multimedia critical editions
    \item object-based argumentation
    \item expanded publication
    \item experiential and spatial
    \item mixed physical and digital
  \end{itemize}
  \item augmented editions and fluid textuality
  \begin{itemize}
    \item structured mark-up
    \item	natural language processing
    \item	relational rhetoric
    \item	textual analysis
    \item	variants and versions
    \item	mutability
  \end{itemize}
  \item scale: the law of large numbers
  \begin{itemize}
    \item quantitative analysis
    \item	text-mining
    \item	machine reading
    \item	digital cultural record
    \item	algorithmic analysis
  \end{itemize}
  \item distant/close, macro/micro, surface/depth
  \begin{itemize}
    \item large-scale patterns
    \item	fine-grained analysis
    \item	close reading
    \item	distant reading
    \item	differential geographies
  \end{itemize}
  \item cultural analytics, aggregation, and data-mining
  \begin{itemize}
    \item parametrics
    \item	cultural mash-ups
    \item	computational processing
    \item	composite analysis
    \item	algorithm design
  \end{itemize}
  \item visualization and data design
  \begin{itemize}
    \item data visualization
    \item	mapping
    \item	information design
    \item	simulation environments
    \item	spatial argument
    \item	modelling knowledge
    \item	visual interpretation
  \end{itemize}
  \item locative investigation and thick mapping
  \begin{itemize}
    \item spatial humanities
    \item	digital cultural mapping
    \item	interconnected sites
    \item	experimental navigation
    \item	geographic information systems (GIS)
    \item	stacked data
  \end{itemize}
  \item the animated archive
  \begin{itemize}
    \item user communities
    \item	permeable walls
    \item	active engagement
    \item	bottom-up curation
    \item	multiplied access
    \item	participatory content creation
  \end{itemize}
  \item distributed knowledge production and performative access
  \begin{itemize}
    \item global networks
    \item	ambient data
    \item	collaborative authorship
    \item	interdisciplinary teams
    \item	use as performance
    \item	crowd-sourcing
  \end{itemize}
  \item humanities gaming
  \begin{itemize}
    \item user engagement
    \item	rule-based play
    \item	rich interaction
    \item	virtual learning environments
    \item	immersion and simulation
    \item	narrative complexity
  \end{itemize}
  \item code, software, and platform studies
  \begin{itemize}
    \item narrative structures
    \item	code as text
    \item	computational processes
    \item	software in a cultural context
    \item	encoding practices
  \end{itemize}
  \item database documentaries
  \begin{itemize}
    \item variable experience
    \item	user-activated
    \item	multimedia prose
    \item	modular and combinatoric
    \item	multilinear
  \end{itemize}
  \item repurposable content and remix culture
  \begin{itemize}
    \item participatory Web
    \item	read/write/rewrite
    \item	platform migration
    \item	sampling and collage
    \item	meta-medium
    \item	inter-textuality
  \end{itemize}
  \item pervasive infrastructure
  \begin{itemize}
    \item extensible frameworks
    \item	heterogeneous data streams
    \item	polymorphous browsing
    \item	cloud computing
  \end{itemize}
  \item ubiquitous scholarship
  \begin{itemize}
    \item augmented reality
    \item	web of things
    \item	pervasive surveillance and tracking
    \item	ubiquitous computing
    \item	deterritorialization of humanistic practice
  \end{itemize}
\end{itemize}
\end{multicols}

\section{Penn Treebank}
\label{s:penntreebank}

\begin{multicols}{2}\raggedright
\begin{description}[leftmargin=1.5cm]
  \item[CC   ] Coordinating conjunction              
  \item[CD   ] Cardinal number                       
  \item[DT   ] Determiner                            
  \item[EX   ] Existential \textit{there}            
  \item[FW   ] Foreign word                          
  \item[IN   ] Preposition/subordinating conjunction 
  \item[JJ   ] Adjective                             
  \item[JJR  ] Adjective, comparative                
  \item[JJS  ] Adjective, superlative                
  \item[LS   ] List item marker                      
  \item[MD   ] Modal                                 
  \item[NN   ] Noun, singular or mass                
  \item[NNS  ] Noun, plural                          
  \item[NNP  ] Proper noun, singular                 
  \item[NNPS ] Proper noun, plural                   
  \item[PDT  ] Predeterminer                         
  \item[POS  ] Possessive ending                     
  \item[PRP  ] Personal pronoun                      
  \item[PP\$ ] Possessive pronoun                    
  \item[RB   ] Adverb                                
  \item[RBR  ] Adverb, comparative                   
  \item[RBS  ] Adverb, superlative                   
  \item[RP   ] Particle                              
  \item[SYM  ] Symbol (mathematical or scientific)   
  \item[TO   ] \textit\{to\}                         
  \item[UH   ] Interjection                          
  \item[VB   ] Verb, base form                       
  \item[VBD  ] Verb, past tense                      
  \item[VBG  ] Verb, gerund/present particle         
  \item[VBN  ] Verb, past particle                   
  \item[VBP  ] Verb, non-3rd ps. sing. present       
  \item[VBZ  ] Verb, 3rd ps. sing. present           
  \item[WDT  ] \textit{wh}-determiner                
  \item[WP   ] \textit{wh}-pronoun                   
  \item[WP\$ ] Possessive \textit{wh}-pronoun        
  \item[WRB  ] \textit{wh}-adverb                    
  \item[\#   ] Pound sign                            
  \item[\$   ] Dollar sign                           
  \item[.    ] Sentence-final punctuation            
  \item[,    ] Comma                                 
  \item[:    ] Colon, semi-colon                     
  \item[(    ] Left bracket character                
  \item[)    ] Right bracket character               
  \item["    ] Straight double quote                 
  \item[`   ] Left open single quote               
  \item[`` ] Left open double quote              
  \item['   ] Right close single quote             
  \item['' ] Right close double quote      
\end{description}      
\end{multicols}


\section{Thinking computers}
\label{s:think}

\begin{quotation}
  \begin{enumerate}
    \item \textbf{Can computers think?}
      \begin{itemize}
        \item Can computers have free will?
        \item Can computers have emotions?
        \item Can computers be creative?
        \item Can computers understand arithmetic?
        \item Can computers draw analogies?
        \item Can computers be persons?
        \item Is the brain a computer?
        \item Can computers reason scientifically?
        \item Are computers inherently disabled?
        \item Should we pretend that computers will never be able to think?
        \item Does God prohibit computers from thinking?
      \end{itemize}
    \item \textbf{Can the Turing test determine whether computers can think?}
      \begin{itemize}
        \item Is failing the test decisive?
        \item Is passing the test decisive?
        \item If a simulated intelligence passes, is it intelligent?
        \item Have any machines passed the test?
        \item Is the test, behaviouraly or operationally construed, a legitimate intelligence test?
        \item Is the test, as a source of inductive evidence, a legitimate intelligence test?
        \item Is the neo-Turing test a legitimate intelligence test?
        \item Does the imitation game determine whether a computer can think?
        \item Can the Loebner Prize stimulate the study of intelligence?
        \item Other Turing test arguments
      \end{itemize}
    \item \textbf{Can physical symbol systems think?}
      \begin{itemize}
        \item Does thinking require a body?
        \item Is the relation between hardware and software similar to that between human brains and minds?
        \item Can physical symbol systems learn as humans do?
        \item Can the elements of thinking be represented in discrete symbolic form?
        \item Can symbolic representations account for human thinking?
        \item Does the situated action paradigm show that computers can't think?
        \item Can physical symbol systems think dialectically?
        \item Can a symbolic knowledge base represent human understanding?
        \item Do humans use rules as physical symbol systems do?
        \item Does mental processing rely on heuristic search?
        \item Do physical symbol systems play chess as humans do?
        \item Other physical system arguments
      \end{itemize}
    \item \textbf{Can Chinese Rooms think?}
      \begin{itemize}
        \item Do humans, unlike computers, have intrinsic intentionality?
        \item Is biological naturalism valid?
        \item Can computers cross the syntax-semantics barrier?
        \item Can learning machines cross the syntax-semantics barrier?
        \item Can brain simulators think?
        \item Can robots think?
        \item Can a combination robot/brain simulator think?
        \item Can the Chinese Room, considered as a total system, think?
        \item Do Chinese Rooms instantiate programs?
        \item Can an internalized Chinese Room think?
        \item Can translations occur between the internalized Chinese Room and the internalizing English speaker?
        \item Can computers have the right causal powers?
        \item Is strong AI a valid category?
        \item Other Chinese Room arguments
      \end{itemize}
    \item \textbf{Can connectionist networks think?}
      \begin{itemize}
        \item Are connectionist networks like human neural networks?
        \item Do connectionist networks follow rules?
        \item Are connectionist networks vulnerable to the arguments against physical symbol systems?
        \item Does the subsymbolic paradigm offer a valid account of connectionism?
        \item Can connectionist networks exhibit systematicity?
        \item Other connectionist arguments
      \end{itemize}
    \item \textbf{Can computers think in images?}
      \begin{itemize}
        \item Can images be realistically be represented in computer arrays?
        \item Can computers represent the analog properties of images?
        \item Can computers recognize Gestalts?
        \item Are images less fundamental than propositions?
        \item Is image psychology a valid approach to mental processing?
        \item Are images quasi-pictorial representations?
        \item Other imagery arguments
      \end{itemize}
    \item \textbf{Do computers have to be conscious to think?}
      \begin{itemize}
        \item Can computers be conscious?
        \item Is consciousness necessary for thought?
        \item Is the consciousness requirement solipsistic?
        \item Can higher-order representations produce consciousness?
        \item Can functional states generate consciousness?
        \item Does physicalism show that computers can be conscious?
        \item Does the connection principle show that consciousness is necessary for thought?
      \end{itemize}
    \item \textbf{Are thinking computers mathematically possible?}
      \begin{itemize}
        \item Is mechanistic philosophy valid?
        \item Does G{\"o}del's theorem show that machines can't think?
        \item Does G{\"o}del's theorem show that machines can't be conscious?
        \item Do mathematical theorems like G{\"o}del's show that computers are intrinsically limited?
        \item Does G{\"o}del's theorem show that mathematical insight is non-algorithmic?
        \item Can automata think?
        \item Is the Lucas argument dialectical?
        \item Can improved machines beat the Lucas argument?
        \item Is the use of consistency in the Lucas argument problematic?
        \item Other Lucas arguments
      \end{itemize}
  \end{enumerate}
  \sourceatright{\autocite{Horn2009}}
\end{quotation}


\section{Oulipo}
\label{s:appoulipo}

\begin{multicols}{3}\raggedright
\begin{itemize}
  \item x mistakes y for z
  \item word ladder
  \item univocalism
  \item transplant
  \item threnodials
  \item tautogram
  \item spoonerism
  \item sonnet
  \item snowball
  \item sestina
  \item septina
  \item rhopalic verse
  \item quenina
  \item pumectation
  \item prisoner's restriction
  \item precooked language
  \item perverse
  \item perverb
  \item permutation
  \item paronomasia
  \item pangram
  \item palindrome
  \item oligogrammatic poem
  \item n + 7
  \item multiple sonnet
  \item multiple-choice narrative
  \item melting snowball
  \item measures
  \item mathews's algorithm
  \item machines for writing
  \item liponymy
  \item lipolexe
  \item lipogram
  \item lescurean permutations
  \item left-handed lipogram
  \item larding
  \item isopangram
  \item isomorphism
  \item isogram
  \item irrational sonnet
  \item inventory
  \item intersective novel
  \item inclusion
  \item imparmigianisation
  \item homovocalism
  \item homosyntactical translation
  \item homosemantic translation
  \item homophony
  \item homophonic translation
  \item homomorphism
  \item homolexical translation
  \item homoconsonantism
  \item holorhyme
  \item heterosexual rhyme
  \item heterogram
  \item haikuisation
  \item grammatical translation
  \item graeco-latin bi-square
  \item franglais
  \item eye-rhyme
  \item equivoque
  \item epithalamium
  \item eodermdrome
  \item end-to-end
  \item elementary morality
  \item eclipse
  \item deunglitsch
  \item delmas's method
  \item definitional literature
  \item cylinder
  \item constraint
  \item combinatorial literature
  \item clinamen
  \item chronogram
  \item chimera
  \item cento
  \item canada dry
  \item braised rhyme
  \item boolean poetry
  \item beautiful outlaw
  \item beautiful in-law
  \item bananagram
  \item axiomatic writing
  \item avalanche
  \item assonance
  \item asphyxiation
  \item arborescent text
  \item aphorism
  \item antonymy
  \item anterhymes
  \item analogue lexicon
  \item anagram
  \item algol poetry
  \item alexandrine
  \item acrostic
  \item acronymic poetry
\end{itemize}
\end{multicols}
