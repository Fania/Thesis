% !TEX root = ../main.tex

\chapter{Random}
\label{app:random}

\vspace{5cm}


\section{Random Sentences}
\label{s:appsentences}

The full list of random sentences used by \url{pata.physics.wtf} for the top and bottom banners on each page (see screenshot~\ref{img:fullscreenshot}).

\begin{itemize}
  \item in posthumous collaboration
  \item the decomposing brain goes on working after death and it is its dreams that are Paradise
  \item plagiarism by anticipation
  \item the applause of silence is the only kind that counts
  \item to understand pataphysics is to fail to understand pataphysics
  \item duration is the transformation of a succession into a reversion
  \item god is the tangential point between zero and infinity
  \item laughter is born out of the discovery of the contradictory
  \item ha ha
  \item the aesthetic of formal constraint
  \item the unique imaginary solution to the absence of problems
  \item the contemporary relationship between science and poetry
  \item a huge and elaborately constructed hoax
  \item only those who attempt the absurd achieve the impossible
  \item the random is opposed to the deterministic
  \item pure multiplicity irreducible to any other sort of unity
  \item persistence and perseverance to buttress a fleeting existence
  \item enfolding a subject laterally, associatively
  \item a doctrine of correspondences, counterpoised by the exotic charm of another system of thought
  \item double negative is necessary to stop the mind believing
  \item the absence of contradictory evidence is not proof of a theory's validity
  \item very wrong in very important ways
  \item no one point of view is final
  \item unification of opposites
  \item an athletic aesthetics of intuitive and instantaneous judgements
  \item constantly diverted from any objective by the very progress which their energy sustains
  \item imagination envisions the reconciliation of the individual with the whole
  \item behind the illusion lies knowledge
  \item a biomolecular bibliomecha of breathtaking beauty
  \item indubitably coherent yet absolutely nonsensical
  \item stylistic and formal experimentation can not be dismissed as purely apolitical
  \item variable phoneme sequences in suspension within a cloud of relative   \item epi-cultural etymologies
  \item speculative solutions for imaginary problems
  \item nonsense is nonsense only when we have not yet found the point of view from which it makes sense
  \item laughter is the discord between tensions being resolved
  \item read with intention to rewrite
  \item a fractal geometry of momentums
  \item minimizing energy, crystallizing latent structure, pleasure is understood as a practice, and accumulates as experience
  \item don't fall out of love
  \item the `something like', the pseudo
  \item the text transforms itself as soon as it is understood
  \item adaptation of archaic mental structures to new environments
  \item a beautifully controlled yet hideously wasteful catastrophe
  \item driven by compulsive urgency to constantly reconceive the whole idea
  \item writing the unrightable wrongs
  \item prisoners of conscience
  \item machina sapiens negotiating the transformation of what is mortal into what is immortal
  \item database hyper-archive applications stimulating relaxation
  \item smelling of the rain it falls on the way down
  \item a thousand ways to greet the dawn
  \item how far up the chain can you put this without ambiguity
  \item a gentle kitten is licking the inside of my heart
  \item was this constrained by you, or restrained by the concept
  \item adjoining always antiquated permutation
  \item joint ventures can go too far away
  \item nowhere to be found here and elsewhere
  \item engender links to a balancing veneration
  \item coerce to do as a sizeable unprocessed primer
  \item plain up be a best concoction in the words
  \item the farcical pandemonium of technology
  \item it is not true that there were any nails
  \item this discovery opens the door into a completely new anti-world
  \item extending as far beyond metaphysics as the latter extends beyond physics
  \item turn the world upside down and inside out
  \item the law of the ascension of a vacuum toward a periphery
  \item the anti-world God not only plays dice, he spells his name backwards
  \item in the absence of a butler, where does the gun fit in
  \item space is defined by simultaneity
  \item time is a flowing stream, a liquid in uniform rectilinear motion
  \item space is a solid, a rigid system of phenomena
  \item the deceleration of our habitual duration conserved by inertia
  \item a perfect elastic solid
  \item movement into the past consists in the perception of the reversibility of phenomena
  \item relativity is absolute
  \item all observations depend on viewpoint and the scale of the scientist
  \item the clinamen, subjective viewpoint and anthopocentrism all rolled into one
  \item the identity of opposites
  \item making negatives do the work of positives
  \item in this year eighteen hundred and ninety-eight
  \item the twenty-seven equivalents
  \item the virgin, the bright, and the beautiful today
  \item the fifth letter of the first word of the first act
  \item voices asymptotic towards death
  \item an epiphenomenon is that which is superinduced upon a phenomenon
  \item concerning the amorphous isle
  \item like soft coral, amoeboid and protoplasmic
  \item searching desperately under the quinuncial trees for the venerable absent one
  \item the night computed its hours
  \item a remarkable epizootic disease
  \item the eternal nothingness
  \item love looks exactly like an iridescent veil and assumes the masked face of a chrysalis
  \item in a telepathic letter
  \item homo est deus
  \item $\infty$ -0 -a + a + 0 = $\infty$
  \item with the aim of computing the qualities of the French
  \item the inferno of subjectivity
\end{itemize}


\section{Heisenberg Quote}
\label{s:heisenberg}

\begin{quotation}
  The overly forceful insistence on the difference between scientific and artistic cognition quite likely derives from the incorrect notion that concepts are firmly attached to `real objects', as if words had a completely clear and definite meaning in their relationship to reality and as if an accurate sentence, constructed from those words, could deliver an intended `objective' factual situation to a more or less absolute degree. But we know, after all, that language too only grasps and shapes reality by turning it into ideas, by idealizing it. Language, too, approaches reality with specific mental forms about which we do not know right away which part of reality they can comprehend and shape. The question about `right' or `wrong' may indeed be rigorously posed and settled within an idealization, but not in relation to reality. That is why the last measure available for scientific knowledge as well is only the degree to which that knowledge is able to illuminate reality or, better, how that illumination allows us `to find our way' better. And who could question that the spiritual content of a work of art too illumines reality for us and makes it translucent? One must come to terms with the fact that only through the process of cognition itself can we determine what we are to understand by `cognition'. That is why any genuine philosophy, too, stands on the threshold between science and poetry. \sourceatright{\autocite{Heisenberg1942}}
\end{quotation}


\section{Digital Humanities Methodology Field Map}
\label{s:dhmap}

The full \textit{Field map of digital humanities: emerging methods and genres} by Burdick et al. \autocite*{Burdick2012}.

\begin{multicols}{2}\raggedright
\begin{itemize}
  \item enhanced critical curation
  \begin{itemize}
    \item digital collections
    \item multimedia critical editions
    \item object-based argumentation
    \item expanded publication
    \item experiential and spatial
    \item mixed physical and digital
  \end{itemize}
  \item augmented editions and fluid textuality
  \begin{itemize}
    \item structured mark-up
    \item	natural language processing
    \item	relational rhetoric
    \item	textual analysis
    \item	variants and versions
    \item	mutability
  \end{itemize}
  \item scale: the law of large numbers
  \begin{itemize}
    \item quantitative analysis
    \item	text-mining
    \item	machine reading
    \item	digital cultural record
    \item	algorithmic analysis
  \end{itemize}
  \item distant/close, macro/micro, surface/depth
  \begin{itemize}
    \item large-scale patterns
    \item	fine-grained analysis
    \item	close reading
    \item	distant reading
    \item	differential geographies
  \end{itemize}
  \item cultural analytics, aggregation, and data-mining
  \begin{itemize}
    \item parametrics
    \item	cultural mash-ups
    \item	computational processing
    \item	composite analysis
    \item	algorithm design
  \end{itemize}
  \item visualization and data design
  \begin{itemize}
    \item data visualization
    \item	mapping
    \item	information design
    \item	simulation environments
    \item	spatial argument
    \item	modelling knowledge
    \item	visual interpretation
  \end{itemize}
  \item locative investigation and thick mapping
  \begin{itemize}
    \item spatial humanities
    \item	digital cultural mapping
    \item	interconnected sites
    \item	experimental navigation
    \item	geographic information systems (GIS)
    \item	stacked data
  \end{itemize}
  \item the animated archive
  \begin{itemize}
    \item user communities
    \item	permeable walls
    \item	active engagement
    \item	bottom-up curation
    \item	multiplied access
    \item	participatory content creation
  \end{itemize}
  \item distributed knowledge production and performative access
  \begin{itemize}
    \item global networks
    \item	ambient data
    \item	collaborative authorship
    \item	interdisciplinary teams
    \item	use as performance
    \item	crowd-sourcing
  \end{itemize}
  \item humanities gaming
  \begin{itemize}
    \item user engagement
    \item	rule-based play
    \item	rich interaction
    \item	virtual learning environments
    \item	immersion and simulation
    \item	narrative complexity
  \end{itemize}
  \item code, software, and platform studies
  \begin{itemize}
    \item narrative structures
    \item	code as text
    \item	computational processes
    \item	software in a cultural context
    \item	encoding practices
  \end{itemize}
  \item database documentaries
  \begin{itemize}
    \item variable experience
    \item	user-activated
    \item	multimedia prose
    \item	modular and combinatoric
    \item	multilinear
  \end{itemize}
  \item repurposable content and remix culture
  \begin{itemize}
    \item participatory Web
    \item	read/write/rewrite
    \item	platform migration
    \item	sampling and collage
    \item	meta-medium
    \item	inter-textuality
  \end{itemize}
  \item pervasive infrastructure
  \begin{itemize}
    \item extensible frameworks
    \item	heterogeneous data streams
    \item	polymorphous browsing
    \item	cloud computing
  \end{itemize}
  \item ubiquitous scholarship
  \begin{itemize}
    \item augmented reality
    \item	web of things
    \item	pervasive surveillance and tracking
    \item	ubiquitous computing
    \item	deterritorialization of humanistic practice
  \end{itemize}
\end{itemize}
\end{multicols}

\section{Penn Treebank}
\label{s:penntreebank}

The full list of \ac{POS} tags mentioned in chapter~\ref{s:pos} \autocite{Marcus1993}.

\begin{multicols}{2}\raggedright
\begin{description}[leftmargin=1.5cm]
  \item[CC   ] Coordinating conjunction              
  \item[CD   ] Cardinal number                       
  \item[DT   ] Determiner                            
  \item[EX   ] Existential \textit{there}            
  \item[FW   ] Foreign word                          
  \item[IN   ] Preposition/subordinating conjunction 
  \item[JJ   ] Adjective                             
  \item[JJR  ] Adjective, comparative                
  \item[JJS  ] Adjective, superlative                
  \item[LS   ] List item marker                      
  \item[MD   ] Modal                                 
  \item[NN   ] Noun, singular or mass                
  \item[NNS  ] Noun, plural                          
  \item[NNP  ] Proper noun, singular                 
  \item[NNPS ] Proper noun, plural                   
  \item[PDT  ] Predeterminer                         
  \item[POS  ] Possessive ending                     
  \item[PRP  ] Personal pronoun                      
  \item[PP\$ ] Possessive pronoun                    
  \item[RB   ] Adverb                                
  \item[RBR  ] Adverb, comparative                   
  \item[RBS  ] Adverb, superlative                   
  \item[RP   ] Particle                              
  \item[SYM  ] Symbol (mathematical or scientific)   
  \item[TO   ] \textit\{to\}                         
  \item[UH   ] Interjection                          
  \item[VB   ] Verb, base form                       
  \item[VBD  ] Verb, past tense                      
  \item[VBG  ] Verb, gerund/present particle         
  \item[VBN  ] Verb, past particle                   
  \item[VBP  ] Verb, non-3rd ps. sing. present       
  \item[VBZ  ] Verb, 3rd ps. sing. present           
  \item[WDT  ] \textit{wh}-determiner                
  \item[WP   ] \textit{wh}-pronoun                   
  \item[WP\$ ] Possessive \textit{wh}-pronoun        
  \item[WRB  ] \textit{wh}-adverb                    
  \item[\#   ] Pound sign                            
  \item[\$   ] Dollar sign                           
  \item[.    ] Sentence-final punctuation            
  \item[,    ] Comma                                 
  \item[:    ] Colon, semi-colon                     
  \item[(    ] Left bracket character                
  \item[)    ] Right bracket character               
  \item["    ] Straight double quote                 
  \item[`   ] Left open single quote               
  \item[`` ] Left open double quote              
  \item['   ] Right close single quote             
  \item['' ] Right close double quote      
\end{description}      
\end{multicols}


\section{Thinking computers}
\label{s:think}

\begin{quotation}
  \begin{enumerate}
    \item \textbf{Can computers think?}
      \begin{itemize}
        \item Can computers have free will?
        \item Can computers have emotions?
        \item Can computers be creative?
        \item Can computers understand arithmetic?
        \item Can computers draw analogies?
        \item Can computers be persons?
        \item Is the brain a computer?
        \item Can computers reason scientifically?
        \item Are computers inherently disabled?
        \item Should we pretend that computers will never be able to think?
        \item Does God prohibit computers from thinking?
      \end{itemize}
    \item \textbf{Can the Turing test determine whether computers can think?}
      \begin{itemize}
        \item Is failing the test decisive?
        \item Is passing the test decisive?
        \item If a simulated intelligence passes, is it intelligent?
        \item Have any machines passed the test?
        \item Is the test, behaviouraly or operationally construed, a legitimate intelligence test?
        \item Is the test, as a source of inductive evidence, a legitimate intelligence test?
        \item Is the neo-Turing test a legitimate intelligence test?
        \item Does the imitation game determine whether a computer can think?
        \item Can the Loebner Prize stimulate the study of intelligence?
        \item Other Turing test arguments
      \end{itemize}
    \item \textbf{Can physical symbol systems think?}
      \begin{itemize}
        \item Does thinking require a body?
        \item Is the relation between hardware and software similar to that between human brains and minds?
        \item Can physical symbol systems learn as humans do?
        \item Can the elements of thinking be represented in discrete symbolic form?
        \item Can symbolic representations account for human thinking?
        \item Does the situated action paradigm show that computers can't think?
        \item Can physical symbol systems think dialectically?
        \item Can a symbolic knowledge base represent human understanding?
        \item Do humans use rules as physical symbol systems do?
        \item Does mental processing rely on heuristic search?
        \item Do physical symbol systems play chess as humans do?
        \item Other physical system arguments
      \end{itemize}
    \item \textbf{Can Chinese Rooms think?}
      \begin{itemize}
        \item Do humans, unlike computers, have intrinsic intentionality?
        \item Is biological naturalism valid?
        \item Can computers cross the syntax-semantics barrier?
        \item Can learning machines cross the syntax-semantics barrier?
        \item Can brain simulators think?
        \item Can robots think?
        \item Can a combination robot/brain simulator think?
        \item Can the Chinese Room, considered as a total system, think?
        \item Do Chinese Rooms instantiate programs?
        \item Can an internalized Chinese Room think?
        \item Can translations occur between the internalized Chinese Room and the internalizing English speaker?
        \item Can computers have the right causal powers?
        \item Is strong AI a valid category?
        \item Other Chinese Room arguments
      \end{itemize}
    \item \textbf{Can connectionist networks think?}
      \begin{itemize}
        \item Are connectionist networks like human neural networks?
        \item Do connectionist networks follow rules?
        \item Are connectionist networks vulnerable to the arguments against physical symbol systems?
        \item Does the subsymbolic paradigm offer a valid account of connectionism?
        \item Can connectionist networks exhibit systematicity?
        \item Other connectionist arguments
      \end{itemize}
    \item \textbf{Can computers think in images?}
      \begin{itemize}
        \item Can images be realistically be represented in computer arrays?
        \item Can computers represent the analog properties of images?
        \item Can computers recognize Gestalts?
        \item Are images less fundamental than propositions?
        \item Is image psychology a valid approach to mental processing?
        \item Are images quasi-pictorial representations?
        \item Other imagery arguments
      \end{itemize}
    \item \textbf{Do computers have to be conscious to think?}
      \begin{itemize}
        \item Can computers be conscious?
        \item Is consciousness necessary for thought?
        \item Is the consciousness requirement solipsistic?
        \item Can higher-order representations produce consciousness?
        \item Can functional states generate consciousness?
        \item Does physicalism show that computers can be conscious?
        \item Does the connection principle show that consciousness is necessary for thought?
      \end{itemize}
    \item \textbf{Are thinking computers mathematically possible?}
      \begin{itemize}
        \item Is mechanistic philosophy valid?
        \item Does G{\"o}del's theorem show that machines can't think?
        \item Does G{\"o}del's theorem show that machines can't be conscious?
        \item Do mathematical theorems like G{\"o}del's show that computers are intrinsically limited?
        \item Does G{\"o}del's theorem show that mathematical insight is non-algorithmic?
        \item Can automata think?
        \item Is the Lucas argument dialectical?
        \item Can improved machines beat the Lucas argument?
        \item Is the use of consistency in the Lucas argument problematic?
        \item Other Lucas arguments
      \end{itemize}
  \end{enumerate}
  \sourceatright{\autocite{Horn2009}}
\end{quotation}


\section{Jarry's Writing}
\label{s:jarry}

A list of Jarry's works in chronological order with their original titles copied from the French Wikipedia entry on Jarry \autocite*{WikiJarry2016}.

\subsubsection{Works}

\begin{itemize}
  \item Les Antliaclastes (1886--1888) poems, reprinted in Ontogénie
  \item La Seconde Vie ou Macaber (1888) reprinted in Les Minutes de sable mémorial
  \item Onénisme ou les Tribulations de Priou (1888) first version of Ubu cocu
  \item Les Alcoolisés (1890) reprinted in les Les Minutes de sable mémorial
  \item Visions actuelles et futures (1894)
  \item ``Haldernablou'' (1894) reprinted in les Les Minutes de sable mémorial
  \item ``Acte unique'' from César-Antéchrist (1894)
  \item Les Minutes de sable mémorial (1894) poems
  \item César Antéchrist (1895)
  \item Ubu roi (1896, version of 1888)
  \item L’Autre Alceste (1896)
  \item Paralipomènes d’Ubu (1896)
  \item Le Vieux de la montagne (1896)
  \item Les Jours et les Nuits (1897), novel
  \item Ubu cocu ou l'Archéoptéryx (1897)
  \item L’Amour en visites (1897, publié en 1898) poems
  \item Gestes et opinions du docteur Faustroll, pataphysicien (achevé en 1898, published in 1911) novel
  \item Petit Almanach (1898)
  \item L’Amour absolu (1899)
  \item Ubu enchaîné (1899, published in 1900)
  \item Messaline (1900)
  \item Almanach illustré du Père Ubu (1901)
  \item ``Spéculations'', in La Revue Blanche (1901)
  \item Le Surmâle (1901, publié en 1902) novel
  \item ``Gestes'' in La Revue Blanche (1901) published in 1969 with ``Spéculations'' in  La Chandelle verte.
  \item L’Objet aimé (1903)
  \item ``Le 14 Juillet'' in Le Figaro (1904)
  \item Pantagruel (1905 opéra-bouffe by Rabelais staged in 1911, music by Claude Terrasse)
  \item Ubu sur la Butte (1906)
  \item Par la taille (1906) opérette
  \item Le Moutardier du pape (1906, publié en 1907) opéra-bouffe
  \item Albert Samain (souvenirs) (1907)
\end{itemize}

\subsubsection{Publications Post-Mortem}

\begin{itemize}
  \item La Dragonne (1907, published in 1943)
  \item Spéculations (1911)
  \item Pieter de Delft (1974) opéra-comique
  \item Jef (1974) play
  \item Le Manoir enchanté (1974) opéra-bouffe staged in 1905
  \item L’Amour maladroit (1974) opérette
  \item Le Bon Roi Dagobert (1974) opéra-bouffe
  \item Léda (1981) opérette-bouffe
  \item Siloques. Superloques. Soliloques Et Interloques De Pataphysique (2001) texts
  \item Paralipomènes d'Ubu/Salle Ubu (2010) livre d'artiste
  \item Ubu marionnette (2010) livre d'artiste
\end{itemize}

\subsubsection{Translations}

\begin{itemize}
  \item La ballade du vieux marin (1893, after The ancient mariner by Coleridge)
  \item Les silènes (1900, translation of German play by Christian Dietrich Grabbe)
  \item Olalla (1901, novel by Stevenson)
  \item La papesse Jeanne (1907, translation of Greek book by d’Emmanuel Rhoïdès)
\end{itemize}

\subsubsection{Contributions}

\begin{multicols}{2}\raggedright
\begin{itemize}
  \item Écho de Paris
  \item L’Art de Paris
  \item Essais d’art libre
  \item Le Mercure de France
  \item La Revue Blanche
  \item Le Livre d’art
  \item La Revue d’art
  \item L’Omnibus de Corinthe
  \item Renaissance latine
  \item Les Marges
  \item La Plume
  \item L'Œil
  \item Le Canard sauvage
  \item Le Festin d'Ésope
  \item Vers et prose
  \item Poésia
  \item Le Critique
\end{itemize}
\end{multicols}


\section{Leary's Tables}
\label{s:leary}

\begin{table}[!htbp]
\caption[Leary's four types of creativity]{Leary's four types of creativity}
\label{tab:Leary1}
  \everyrow{\hrule}
  \tabulinesep = 2mm
  \begin{tabu}{|X[L]|X[L]|X[L]|X[L]|}
  \textbf{Reproductive Blocked}
  &
  \textbf{Reproductive Creator}
  &
  \textbf{Creative Creator}
  &
  \textbf{Creative Blocked}
  \\
  The routine, well-socialised person who experiences only in terms of what he has been taught and who produces only what has been produced before.
  &
  The innovating performer who experiences only in terms of the available categories but has learned to manipulate these categories in novel combinations.
  &
  The person who experiences directly outside the limits of ego and labels, and who has learned to develop new models of communications, or who can manipulate familiar categories in novel combinations or who can let natural modes develop under his nurture.
  &
  The person who experiences uniquely and sensitively outside of game concepts (either by choice or helplessly by inability) but who is unable to communicate or uninterested in communicating these experiences outside the conventional manner.
  \\
  Reproductive Performer
  &
  \multicolumn{2}{c|}{Creative Performer}
  &
  Reproductive Performer
  \\
  \multicolumn{2}{|c|}{Reproductive Experience}
  &
  \multicolumn{2}{c|}{Creative Experience}
  \\
  \end{tabu}
\end{table}

\begin{table}[!htbp]
\caption[Leary's social labels]{Leary's social labels to describe the types of creativity}
\label{tab:Leary2}
  \everyrow{\hrule}
  \tabulinesep = 2mm % chktex 1
  \begin{tabu}{|X[L]|X[L]|X[L]|X[L]|}
  \textbf{Reproductive Blocked}
  &
  \textbf{Reproductive Creator}
  &
  \textbf{Creative Creator}
  &
  \textbf{Creative Blocked}
  \\
  Unimaginative, incompetent hack.
  &
  Reliable nihilist, insensitive, unsuccessful innovator whose shock value changes to morbid curiosity as fads of performance change.
  &
  The mad creative genius, the undiscovered far-out crackpot creator who is recognised by later generations as a creative giant.
  &
  Psychotic, religious crank, eccentric who uses conventional forms for expressing mystical convictions.
  \\
  Competent, responsible, reliable worker.
  &
  Bold initiator who wins game recognitions but whose fame crumbles as fads of performance change.
  &
  The truly creative giant recognised by his own age and the ages to come.
  &
  Solid, reliable person with a `deep streak'.
  \\
  Reproductive Performer
  &
  \multicolumn{2}{c|}{Creative Performer}
  &
  Reproductive Performer
  \\
  \multicolumn{2}{|c|}{Reproductive Experience}
  &
  \multicolumn{2}{c|}{Creative Experience}
  \\
  \end{tabu}
\end{table}