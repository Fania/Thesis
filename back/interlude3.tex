% !TEX root = ../main.tex

\pagestyle{empty}

\chapter*{INTERLUDE III}
\label{interlude3}

% analysis
% aspirations
% observations


\begin{quotation}
  There is no pure science of creativity, because it is paradigmatically idiographic --- it can only be understood against the backdrop of a particular history.\sourceatright{\autocite{Elton1995}}
\end{quotation}

\begin{quotation}
  Tools are not just tools. They are cognitive interfaces that presuppose forms of mental and physical discipline and organization. By scripting an action, they produce and transmit knowledge, and, in turn, model a world. \sourceatright{\autocite[p.105]{Burdick2012}}
\end{quotation}

\begin{quotation}
  Humanists have begun to use programming languages. But they have yet to create programming languages of their own: languages that can come to grips with, for example, such fundamental attributes of cultural communication and traditional objects of humanistic scrutiny as nuance, inflection, undertone, irony, and ambivalence. \sourceatright{\autocite[p.103]{Burdick2012}}
\end{quotation}

\pagestyle{fania}


\clearpage