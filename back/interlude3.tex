% !TEX root = ../main.tex

\pagestyle{empty}

\chapter*{INTERLUDE III}
\label{interlude3}

% analysis
% aspirations
% observations


\begin{quotation}
  There is no pure science of creativity, because it is paradigmatically idiographic --- it can only be understood against the backdrop of a particular history.\sourceatright{\autocite{Elton1995}}
\end{quotation}

\begin{quotation}
  Evaluation is thus a matter of subjectivity, since no scientism allows us to pretend to objectivity, an objectivity aspired to on the illusory grounds that it would support taking a decision without the decision-maker simultaneously taking a risk or responsibility. \sourceatright{\autocite[Montfort and deVarine, cited in][Matarasso's translation]{Matarasso1997}}
\end{quotation}

\begin{quotation}
  Tools are not just tools. They are cognitive interfaces that presuppose forms of mental and physical discipline and organization. By scripting an action, they produce and transmit knowledge, and, in turn, model a world. \sourceatright{\autocite{Burdick2012}}
\end{quotation}

\begin{quotation}
  Humanists have begun to use programming languages. But they have yet to create programming languages of their own: languages that can come to grips with, for example, such fundamental attributes of cultural communication and traditional objects of humanistic scrutiny as nuance, inflection, undertone, irony, and ambivalence. \sourceatright{\autocite{Burdick2012}}
\end{quotation}

\begin{quotation}
  Conceptually, I'm curious about what happens when an algorithm passes the uncanny valley and becomes a perfect mimic. If humans were unable to distinguish the generated drug experience from a real one, the machine would become a sort of philosophical zombie: an entity that appears to be something that it isn't, something it could never be. \sourceatright{\autocite{McDonald2016}}
\end{quotation}


\pagestyle{fania}


\clearpage