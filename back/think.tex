% !TEX root = ../main.tex

\chapter{Thinking computers}
\label{app:think}


\begin{quotation}
  \begin{enumerate}
    \item \textbf{Can computers think?}
      \begin{itemize}
        \item Can computers have free will?
        \item Can computers have emotions?
        \item Can computers be creative?
        \item Can computers understand arithmetic?
        \item Can computers draw analogies?
        \item Can computers be persons?
        \item Is the brain a computer?
        \item Can computers reason scientifically?
        \item Are computers inherently disabled?
        \item Should we pretend that computers will never be able to think?
        \item Does God prohibit computers from thinking?
      \end{itemize}
    \item \textbf{Can the Turing test determine whether computers can think?}
      \begin{itemize}
        \item Is failing the test decisive?
        \item Is passing the test decisive?
        \item If a simulated intelligence passes, is it intelligent?
        \item Have any machines passed the test?
        \item Is the test, behaviouraly or operationally construed, a legitimate intelligence test?
        \item Is the test, as a source of inductive evidence, a legitimate intelligence test?
        \item Is the neo-Turing test a legitimate intelligence test?
        \item Does the imitation game determine whether a computer can think?
        \item Can the Loebner Prize stimulate the study of intelligence?
        \item Other Turing test arguments
      \end{itemize}
    \item \textbf{Can physical symbol systems think?}
      \begin{itemize}
        \item Does thinking require a body?
        \item Is the relation between hardware and software similar to that between human brains and minds?
        \item Can physical symbol systems learn as humans do?
        \item Can the elements of thinking be represented in discrete symbolic form?
        \item Can symbolic representations account for human thinking?
        \item Does the situated action paradigm show that computers can't think?
        \item Can physical symbol systems think dialectically?
        \item Can a symbolic knowledge base represent human understanding?
        \item Do humans use rules as physical symbol systems do?
        \item Does mental processing rely on heuristic search?
        \item Do physical symbol systems play chess as humans do?
        \item Other physical system arguments
      \end{itemize}
    \item \textbf{Can Chinese Rooms think?}
      \begin{itemize}
        \item Do humans, unlike computers, have intrinsic intentionality?
        \item Is biological naturalism valid?
        \item Can computers cross the syntax-semantics barrier?
        \item Can learning machines cross the syntax-semantics barrier?
        \item Can brain simulators think?
        \item Can robots think?
        \item Can a combination robot/brain simulator think?
        \item Can the Chinese Room, considered as a total system, think?
        \item Do Chinese Rooms instantiate programs?
        \item Can an internalized Chinese Room think?
        \item Can translations occur between the internalized Chinese Room and the internalizing English speaker?
        \item Can computers have the right causal powers?
        \item Is strong AI a valid category?
        \item Other Chinese Room arguments
      \end{itemize}
    \item \textbf{Can connectionist networks think?}
      \begin{itemize}
        \item Are connectionist networks like human neural networks?
        \item Do connectionist networks follow rules?
        \item Are connectionist networks vulnerable to the arguments against physical symbol systems?
        \item Does the subsymbolic paradigm offer a valid account of connectionism?
        \item Can connectionist networks exhibit systematicity?
        \item Other connectionist arguments
      \end{itemize}
    \item \textbf{Can computers think in images?}
      \begin{itemize}
        \item Can images be realistically be represented in computer arrays?
        \item Can computers represent the analog properties of images?
        \item Can computers recognize Gestalts?
        \item Are images less fundamental than propositions?
        \item Is image psychology a valid approach to mental processing?
        \item Are images quasi-pictorial representations?
        \item Other imagery arguments
      \end{itemize}
    \item \textbf{Do computers have to be conscious to think?}
      \begin{itemize}
        \item Can computers be conscious?
        \item Is consciousness necessary for thought?
        \item Is the consciousness requirement solipsistic?
        \item Can higher-order representations produce consciousness?
        \item Can functional states generate consciousness?
        \item Does physicalism show that computers can be conscious?
        \item Does the connection principle show that consciousness is necessary for thought?
      \end{itemize}
    \item \textbf{Are thinking computers mathematically possible?}
      \begin{itemize}
        \item Is mechanistic philosophy valid?
        \item Does G{\"o}del's theorem show that machines can't think?
        \item Does G{\"o}del's theorem show that machines can't be conscious?
        \item Do mathematical theorems like G{\"o}del's show that computers are intrinsically limited?
        \item Does G{\"o}del's theorem show that mathematical insight is non-algorithmic?
        \item Can automata think?
        \item Is the Lucas argument dialectical?
        \item Can improved machines beat the Lucas argument?
        \item Is the use of consistency in the Lucas argument problematic?
        \item Other Lucas arguments
      \end{itemize}
  \end{enumerate}
  \sourceatright{\autocite{Horn2009}}
\end{quotation}