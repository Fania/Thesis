% !TEX root = ../main.tex

\chapter{Others}
\label{app:others}


\section{Digital Humanities Methodology Field Map}
\label{s:dhmap}

The full \textit{Field map of digital humanities: emerging methods and genres} by Burdick et al. \citeyear[p.29-60]{Burdick2012}.

\begin{itemize}
  \item enhanced critical curation
  \begin{itemize}
    \item digital collections
    \item multimedia critical editions
    \item object-based argumentation
    \item expanded publication
    \item experiential and spatial
    \item mixed physical and digital
  \end{itemize}
  \item augmented editions and fluid textuality
  \begin{itemize}
    \item structured mark-up
    \item	natural language processing
    \item	relational rhetoric
    \item	textual analysis
    \item	variants and versions
    \item	mutability
  \end{itemize}
  \item scale: the law of large numbers
  \begin{itemize}
    \item quantitative analysis
    \item	text-mining
    \item	machine reading
    \item	digital cultural record
    \item	algorithmic analysis
  \end{itemize}
  \item distant/close, macro/micro, surface/depth
  \begin{itemize}
    \item large-scale patterns
    \item	fine-grained analysis
    \item	close reading
    \item	distant reading
    \item	differential geographies
  \end{itemize}
  \item cultural analytics, aggregation, and data-mining
  \begin{itemize}
    \item parametrics
    \item	cultural mash-ups
    \item	computational processing
    \item	composite analysis
    \item	algorithm design
  \end{itemize}
  \item visualization and data design
  \begin{itemize}
    \item data visualization
    \item	mapping
    \item	information design
    \item	simulation environments
    \item	spatial argument
    \item	modelling knowledge
    \item	visual interpretation
  \end{itemize}
  \item locative investigation and thick mapping
  \begin{itemize}
    \item spatial humanities
    \item	digital cultural mapping
    \item	interconnected sites
    \item	experimental navigation
    \item	geographic information systems (GIS)
    \item	stacked data
  \end{itemize}
  \item the animated archive
  \begin{itemize}
    \item user communities
    \item	permeable walls
    \item	active engagement
    \item	bottom-up curation
    \item	multiplied access
    \item	participatory content creation
  \end{itemize}
  \item distributed knowledge production and performative access
  \begin{itemize}
    \item global networks
    \item	ambient data
    \item	collaborative authorship
    \item	interdisciplinary teams
    \item	use as performance
    \item	crowd-sourcing
  \end{itemize}
  \item humanities gaming
  \begin{itemize}
    \item user engagement
    \item	rule-based play
    \item	rich interaction
    \item	virtual learning environments
    \item	immersion and simulation
    \item	narrative complexity
  \end{itemize}
  \item code, software, and platform studies
  \begin{itemize}
    \item narrative structures
    \item	code as text
    \item	computational processes
    \item	software in a cultural context
    \item	encoding practices
  \end{itemize}
  \item database documentaries
  \begin{itemize}
    \item variable experience
    \item	user-activated
    \item	multimedia prose
    \item	modular and combinatoric
    \item	multilinear
  \end{itemize}
  \item repurposable content and remix culture
  \begin{itemize}
    \item participatory Web
    \item	read/write/rewrite
    \item	platform migration
    \item	sampling and collage
    \item	meta-medium
    \item	inter-textuality
  \end{itemize}
  \item pervasive infrastructure
  \begin{itemize}
    \item extensible frameworks
    \item	heterogeneous data streams
    \item	polymorphous browsing
    \item	cloud computing
  \end{itemize}
  \item ubiquitous scholarship
  \begin{itemize}
    \item augmented reality
    \item	web of things
    \item	pervasive surveillance and tracking
    \item	ubiquitous computing
    \item	deterritorialization of humanistic practice
  \end{itemize}
\end{itemize}