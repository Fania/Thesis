% !TEX root = ../main.tex

\pagestyle{empty}

\chapter*{INTERLUDE II}
\label{interlude2}

% foundation
% interpretation
% implementation
% application


\begin{quotation}
  all the familiar landmarks of my thought - our thought, the thought that bears the stamp of our age and our geography - breaking up all the ordered surfaces and all the planes with which we are accustomed to tame the wild profusion of existing things, and continuing long afterwards to disturb and threaten with collapse our age-old distinction between the Same and the Other. \sourceatright{\autocite{Foucault1966}---taking about Borges}
\end{quotation}

\begin{quotation}
    Only those who attempt the absurd achieve the impossible. \sourceatright{(attributed to M.C. Escher)}
\end{quotation}

\begin{quotation}
    A great truth is a truth whose opposite is also a great truth. Thomas Mann \sourceatright{\autocite[as cited in][]{Wickson2006}}
\end{quotation}

\begin{quotation}
    Heisenberg's Uncertainty Principle is merely an application, a demonstration of the Clinamen, subjective viewpoint and anthropocentrism all rolled into one. \sourceatright{\autocite{Jarry2006}}
\end{quotation}

\begin{quotation}
    Epiphany – `to express the bursting forth or the revelation of pataphysics' \sourceatright{Dr Sandomir \autocite[p.174]{Hugill2012a}}
\end{quotation}

\begin{quotation}
    Machines take me by surprise with great frequency.\sourceatright{\autocite[p.54]{Turing2009}}
\end{quotation}

\begin{quotation}
    The view that machines cannot give rise to surprises is due, I believe, to a fallacy to which philosophers and mathematicians are particularly subject. This is the assumption that as soon as a fact is presented to a mind all consequences of that fact spring into the mind simultaneously with it.\sourceatright{\autocite[p.54]{Turing2009}}
\end{quotation}

\begin{quotation}
  Opposites are complementary.\\
  It is the hallmark of any deep truth that its negation is also a deep truth.\\
  Some subjects are so serious that one can only joke about them.
  \sourceatright{Niels Bohr}
\end{quotation}

\clearpage