% !TEX root = ../main.tex

\pagestyle{empty}

\chapter*{INTERLUDE II}
\label{interlude2}

% foundation
% interpretation
% implementation
% application

\begin{quotation}
  Computation is not a fact of nature. It's a fact of our interpretation.\sourceatright{\autocite{Searle2015}}
\end{quotation}

\begin{quotation}
  all the familiar landmarks of my thought - our thought, the thought that bears the stamp of our age and our geography - breaking up all the ordered surfaces and all the planes with which we are accustomed to tame the wild profusion of existing things, and continuing long afterwards to disturb and threaten with collapse our age-old distinction between the Same and the Other. \sourceatright{\autocite{Foucault1966}---taking about Borges}
\end{quotation}

\begin{quotation}
    Only those who attempt the absurd achieve the impossible. \sourceatright{(attributed to M.C. Escher)}
\end{quotation}

\begin{quotation}
    A great truth is a truth whose opposite is also a great truth. Thomas Mann \sourceatright{\autocite[as cited in][]{Wickson2006}}
\end{quotation}

\begin{quotation}
    Heisenberg's Uncertainty Principle is merely an application, a demonstration of the Clinamen, subjective viewpoint and anthropocentrism all rolled into one. \sourceatright{\autocite{Jarry2006}}
\end{quotation}

\begin{quotation}
    Epiphany – `to express the bursting forth or the revelation of pataphysics' \sourceatright{Dr Sandomir \autocite[p.174]{Hugill2012a}}
\end{quotation}

\begin{quotation}
    Machines take me by surprise with great frequency.\sourceatright{\autocite[p.54]{Turing2009}}
\end{quotation}

\begin{quotation}
    The view that machines cannot give rise to surprises is due, I believe, to a fallacy to which philosophers and mathematicians are particularly subject. This is the assumption that as soon as a fact is presented to a mind all consequences of that fact spring into the mind simultaneously with it.\sourceatright{\autocite[p.54]{Turing2009}}
\end{quotation}

\begin{quotation}
  Opposites are complementary.\\
  It is the hallmark of any deep truth that its negation is also a deep truth.\\
  Some subjects are so serious that one can only joke about them.
  \sourceatright{Niels Bohr}
\end{quotation}

\begin{quotation}
  There is no pure science of creativity, because it is paradigmatically idiographic --- it can only be understood against the backdrop of a particular history.\sourceatright{\autocite{Elton1995}}
\end{quotation}

\begin{quote}
  Tools are not just tools. They are cognitive interfaces that presuppose forms of mental and physical discipline and organization. By scripting an action, they produce and transmit knowledge, and, in turn, model a world. \sourceatright{\autocite[p.105]{Burdick2012}}
\end{quote}

\begin{quote}
  Humanists have begun to use programming languages. But they have yet to create programming languages of their own: languages that can come to grips with, for example, such fundamental attributes of cultural communication and traditional objects of humanistic scrutiny as nuance, inflection, undertone, irony, and ambivalence. \sourceatright{\autocite[p.103]{Burdick2012}}
\end{quote}

% \begin{draft}
%   ``Since our solutions will be imaginary, our aim is not so much to have the computer generate creative artefacts as to engage in a creative dialogue with the user. Therefore, we do not intend to move as close to artificial intelligence as Colton's framework seems to suggest \autocite{Colton2008}. In the pataphysical universe, ideas such as `human skill', `human imagination' and `human appreciation' are too generalised to be useful. One may very well ask: which human? And when, where and even why? Rather, our project will aim to produce an exceptional computational entity that consistently generates surprising and novel provocations to the users, who in turn may navigate and modify these by deploying their own skills, appreciation and imagination. The relationship between the two will develop quite rapidly into one of mutual subversion since, however apparent the `rules of the game' may become, the outcomes will always be particular or exceptional.'' \autocite{Hugill2013d}
% \end{draft}

\clearpage