\marginpar{§\ref{ch:creativity}}


\todo{golden compass, paulman?}

% marvosym
\Forward
\MoveDown
\RewindToIndex
\ToTop
\ForwardToEnd
\MoveUp
\RewindToStart
\ForwardToIndex
\Rewind
\ToBottom

% bclogo
\bcloupe

% fontawesome
\faicon{search}
\faicon{searchplus}
\faicon{info}
\faicon{comment}
\faicon{commentO}
\faicon{eye}
\faicon{table}

@book{Hugill2012,
  author = {Hugill, Andrew},
  booktitle = {Philosophy},
  publisher = {Self},
  title = {{Lineaments of 'Pataphysics}},
  year = {2012}
}


``These types of results are incredably useful for any one who derives value from new ideas.''\autocite{Yossarian2015}

In regards to my project:
\begin{itemize}
  \item A concept implementation method is used with a descriptive-other approach
  \item A qualitative investigation into if and why the proposed search results are useful will be done
  \item Following experimental methodologies, to evaluate the proposed new solution to the problem of creative search
\end{itemize}


bridge: how do current search engines work? they prioritise revelvance using pagerank algorithms etc. happens at crawling time. pataphysics isnt about relevance. (index is ranked)

pataphysics cant be ranked. need for neutrality in index but creative ways to retrieve matches for query.
but then changed to focus on the concept of searching/browsing (in itself, rather than part of a system architecture) and ranking as a creative process.
pataphysicalisation happens at query time between query and index. (index is neutral)

project was to build a prototype that proves these ideas.
my eventual approach was to take elemnts from both the ontology idea and the relevance ranking IR way and combine/redeploy them in a new way using pataphysics that would yield results designed to foster/inspire creativity.



\begin{fcom}
  poems = combinatorial creativity visualised.
  From calculated relevance to creative detour.
\end{fcom}

\begin{fcom}
  noise\\
  basins of attraction\\
  break symmetry\\
  creativity is not linear\\
  phase shift\\
  \autocite{Everitt2011}
\end{fcom}

\grule
