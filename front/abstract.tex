% !TEX root = ../main.tex

\pagestyle{empty}

\chapter{TL;DR}
\label{abstract}

{\Large \textbf{Algorithmic Meta-Creativity}} --- Fania Raczinski

\vspace{0.5cm}
ABSTRACT\footnote{``Too long; didn't read''} --- 300 words

\begin{draft}
  Using computers to produce creative artefacts is a form of computational creativity. Using creative techniques computationally is creative computing. \gls{amc} spans the two---whether this is to achieve a creative or non-creative output. It is the use of digital tools (which may not be creative themselves) and the way they are used forms the creative process or product. 

  Creativity in humans needs to be interpreted differently to machines. Humans and machines differ in many ways, we have different `brains/memory', `thinking processes/software' and `bodies/hardware'. Too often creative output by machines is judged as we would a humans. 

  Computers which are truly artificially intelligent might be capable of true artificial creativity. Until then they are (philosophical) zombie robots: machines that behave like humans but aren't conscious. The only alternative is to see any computer creativity as an expression of human creativity using digital means and evaluate it as such.

  \gls{amc} is neither machine creativity nor human creativity---it is both.

  By acknowledging the undeniable link between computer creativity and its human influence (the machine is just a tool for the human) we enter a new realm of thought. How is \gls{amc} defined and evaluated?

  This thesis address this issue. First a practical demonstration of \gls{amc} is presented (\url{pata.physics.wtf}) and then a theoretical framework to help interpret and evaluate products of \gls{amc} is explained.
\end{draft}

\textit{\textbf{Keywords:} Algorithmic Meta-Creativity, Creative computing, Pataphysics, Computational Creativity, Creativity}

\clearpage
