% !TEX root = ../main.tex

\pagestyle{empty}

\chapter{Abstract}
\label{abstract}

{\Large \textbf{Web Collider - Fania Raczinski}}

\vspace{0.5cm}

A pataphysical methodology for applying creativity to exploratory search

\vspace{1cm}

Obscure French Philosophy

This paper looks at defining, analysing and practicing how creativity can be applied to search tools. It defines creativity with respect to search and discusses how these concepts could be applied in software engineering using principles from the pseudo-philosophy of pataphysics. The aim of the proposed tool is to generate surprising, novel, humorous and provocative search results instead of purely relevant ones, in order to inspire a more creative interaction between a user, their information need and the application. A proof-of-concept prototype is described to justify the ideas presented before issues and future work are discussed.

We introduce the idea of a new kind of Web search tool that uses the literary and philosophical idea of pataphysics as a conceptual framework in order to return creative results. Pataphysics, the science of exceptions and imaginary solutions, can be directly linked to creativity and is therefore very suitable to guide the transformation from relevant into creative search results. To enable pataphysical algorithms within our system we propose the need for a new type of system architecture. We discuss a component-based software architecture that would allow the flexible integration of the new algorithms at any stage or location and the need for an index suitable to handle patadata, data which has been transformed pataphysically. This tool aims to generate surprising, novel and provocative search results rather than relevant ones, in order to inspire a more creative interaction that has applications in both creative work and learning contexts.

\clearpage
