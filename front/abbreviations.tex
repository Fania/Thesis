% !TEX root = ../main.tex

\makenoidxglossaries

% Here’s my \gls{hyper} term. \gls{holo} \gls{hyper}\\
% First use: \gls{ir}. Second use: \gls{ir}.\\
% First use: \gls{nlp}. Second use: \gls{nlp}.

\newacronym{ir}{IR}{Information Retrieval}
\newacronym{nlp}{NLP}{Natural Language Processing}
\newacronym{in}{IN}{Information Need}
\newacronym{ui}{UI}{User Interface}
\newacronym{svm}{SVM}{Support Vector Machine}
\newacronym{ai}{AI}{Artificial Intelligence}
\newacronym{iccc}{ICCC}{International Conference on Computational Creativity}
\newacronym{compc}{CompC}{Computational Creativity}
\newacronym{cc}{CC}{Creative Computing}
\newacronym{ijcrc}{IJCrC}{International Journal of Creative Computing}
\newacronym{dh}{DH}{Digital Humanities}



\newglossaryentry{hyper}{name={hypernym},
description={A hyponym shares a type-of relationship with its hypernym. For example, pigeon, crow, eagle and seagull are all hyponyms of bird (their hypernym); which, in turn, is a hyponym of animal.}
}

\newglossaryentry{holo}{name={holonym},
description={The relationship between a term denoting the whole and a term denoting a part of, or a member of, the whole. That is, 'X' is a holonym of 'Y' if Ys are parts of Xs, or 'X' is a holonym of 'Y' if Ys are members of Xs. For example, 'tree' is a holonym of 'bark', of 'trunk' and of 'limb.' Holonymy is the opposite of meronymy.}
}

\newglossaryentry{index}{name={index},
description={Lookup table, a data structure, usually an array or associative array, often used to replace a runtime computation with a simpler array indexing operation.}
}

\newglossaryentry{bisociation}{name={bisociation},
description={Two self-consistent but habitually incompatible frames of reference intersecting to give rise to a new creative idea.}
}
