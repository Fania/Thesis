% !TEX root = ../main.tex

\makenoidxglossaries{}

% Here’s my \gls{hyper} term. \gls{holo} \gls{hyper}\\
% First use: \gls{ir}. Second use: \gls{ir}.\\
% First use: \gls{nlp}. Second use: \gls{nlp}.
% \Gls{xyz}: capitalise
% \glspl{xyz}: pluralise
% \acrlong{<label>}
% \acrfull{<label>}
% \acrshort{<label>}

% \glossary[hfilei](hkeyi){htermi}{hdesci}

\newacronym{amc}{AMC}{Algorithmic Meta-Creativity}
\newacronym{ir}{IR}{Information Retrieval}
\newacronym{nlp}{NLP}{Natural Language Processing}
\newacronym{in}{IN}{Information Need}
\newacronym{ui}{UI}{User Interface}
\newacronym{svm}{SVM}{Support Vector Machine}
\newacronym{ai}{AI}{Artificial Intelligence}
\newacronym{iccc}{ICCC}{International Conference on Computational Creativity}
\newacronym{compc}{CompC}{Computational Creativity}
\newacronym{cc}{CC}{Creative Computing}
\newacronym{ijcrc}{IJCrC}{International Journal of Creative Computing}
\newacronym{dh}{DH}{Digital Humanities}
\newacronym{nltk}{NLTK}{Natural Language Tool Kit}
\newacronym{tf}{TF}{Term Frequency}
\newacronym{idf}{IDF}{Inverse Document Frequency}
\newacronym{tdm}{TDM}{Term-Document Matrix}
\newacronym{api}{API}{Application Program Interface}
\newacronym{rest}{REST}{Representational State Transfer}
\newacronym{http}{HTTP}{Hypertext Transfer Protocol}
\newacronym{ioct}{IOCT}{Institute of Creative Technologies}
\newacronym{lms}{LMS}{Leicester Media School}
\newacronym{dmu}{DMU}{De Montfort University}
\newacronym{cas}{CAS}{Computer Arts Society}
\newacronym{tdc}{TDC}{Transdisciplinary Common Room}
\newacronym{url}{URL}{Uniform Resource Locator}
\newacronym{json}{JSON}{JavaScript Object Notation}
\newacronym{tmpr}{TMPR}{Trajectory Model of Practice and Research}
\newacronym{html}{HTML}{Hypertext Markup Language}
\newacronym{css}{CSS}{Cascading Stylesheets}
\newacronym{hci}{HCI}{Human Computer Interaction}
\newacronym{mmce}{MMCE}{Multi-dimensional Model of Creativity and Evaluation}
\newacronym{specs}{SPECS}{Standardised Procedure for Evaluating Creative Systems}
\newacronym{csf}{CSF}{Creative Search Framework}
\newacronym{pep}{PEP}{Python Enhancement Proposal}
\newacronym{bdfl}{BDFL}{Benevolent Dictator For Life}
\newacronym{ioccc}{IOCCC}{International Obfuscated C Code Contest}





\newglossaryentry{hyper}{name={hypernym},
description={A hyponym shares a type-of relationship with its hypernym. For example, pigeon, crow, eagle and seagull are all hyponyms of bird (their hypernym); which, in turn, is a hyponym of animal.}
}

\newglossaryentry{holo}{name={holonym},
description={The relationship between a term denoting the whole and a term denoting a part of, or a member of, the whole. That is, `X' is a holonym of `Y' if Ys are parts of Xs, or `X' is a holonym of `Y' if Ys are members of Xs. For example, `tree' is a holonym of `bark', of `trunk' and of `limb.' Holonymy is the opposite of meronymy.}
}

\newglossaryentry{index}{name={index},
description={Lookup table, a data structure, usually an array or associative array, often used to replace a runtime computation with a simpler array indexing operation.}
}

\newglossaryentry{bisociation}{name={bisociation},
description={Two self-consistent but habitually incompatible frames of reference intersecting to give rise to a new creative idea.}
}

\newglossaryentry{get}{name={GET},
description={An \gls{http} method. Allows a client (browser) to request data from a specified resource on a given web server.}
}

\newglossaryentry{post}{name={POST},
description={An \gls{http} method. Allows a client (browser) to submit data to be processed to a specified resource on a given web server.}
}
