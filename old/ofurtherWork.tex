% !TEX root = ../main.tex

\pagestyle{fancy}

\chapter{Further Work}
\label{FurtherWork}
\lhead{Chapter 8. \emph{Further Work}}

%------------------------------------------------------------------------------

\section{Intro}

From here, we can try to implement different algorithms or different pataphysical concepts within our existing tool or built a different system. The next logical step would be to implement a fully functioning Web search engine using the algorithms described in this paper. But before we go into further development, it might be worth evaluating and interpreting the results produced by the prototype.

In this paper we have introduced a new approach for a creative search tool that uses pataphysics as an underlying philosophy.  We have explained how pataphysics can be used in search algorithms to produce interesting results with a humorous twist. Our initial experiments within a limited domain have shown that the generated results can indeed be novel, surprising and useful. We have also briefly discussed ideas for applications of the tool and issues that may trigger possible further research in in the field of Computing. We have presented some thoughts on evaluation of our tool and future work.

5.1	FURTHER DEBUGGING OF CODE (IF NECESSARY)

5.2	IMPROVEMENTS/ALTERNATIVES TO USER INTERFACE DESIGN

5.3	IMPROVEMENTS/ALTERNATIVES TO ALGORITHMS

5.4	IMPROVEMENTS/ALTERNATIVES TO ARCHITECTURE
 
%------------------------------------------------------------------------------

\section{Code}

Sed ullamcorper quam eu nisl interdum at interdum enim egestas.

%------------------------------------------------------------------------------

\section{Interface}

%------------------------------------------------------------------------------

\section{Algorithms}

%------------------------------------------------------------------------------

\section{Architecture}
