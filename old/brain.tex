
-----------------------------

\begin{fcom}
•	Brain operations per sec 1016 \autocite[p.194]{Kurzweil2013}\\
•	Japan’s K-computer has 1016 calculations per sec (10 petaflops)\\
•	Blue brain project: 2023: 1017 bytes memory + 1018 flops \autocite[p.125]{Kurzweil2013}
\end{fcom}

Human Brain Project: \autocite{Walker2012}

Our brain consumes about 30W, the same as an electric light bulb, thousands of times less than a small supercomputer. \autocite[p.17]{Walker2012}

For environmental and business reasons, vendors have set themselves the goal of containing energy consumption to a maximum of 20 megawatts  \autocite[p.41]{Walker2012}

the 1 PFlop machine at the Jülich Supercomputing Centre could simulate up to 100 million neurons – roughly the number found in the mouse brain. \autocite[p.41]{Walker2012}

Cellular-level simulation of the 100 billion neurons of the human brain will require compute power at the exascale (1018 flops). \autocite[p.41-42]{Walker2012}

2017 petascale 50petabytes memory + 50 petaflops + <=4MW power

2021 exascale 200petabyte memory + 1exaflop

A second, equally important goal will be to prepare the procurement of the HBP Pre-exascale-supercomputer. By 2017/18, Jülich plans to procure a Big Data-centred system with at least 50 PBytes of hierarchical storage-class memory, a peak capability of at least 50 PFlop/s and a power consumption <= 4 MW. The memory and computational speed of the machine will be sufficient to simulate a realistic mouse brain and to develop first-draft models of the human brain. (The rest of the hardware roadmap targets an exascale machine in 2021/2022 with a capability of 1 EFlop/s and a hierarchical storage-class memory of 200 PB).\footnote{https://www.humanbrainproject.eu/high-performance-computing-platform}

Chris Chatham: 10 Important Differences Between Brains and Computers \footnote{http://scienceblogs.com/developingintelligence/2007/03/27/why-the-brain-is-not-like-a-co/}

\begin{enumerate}
\item Brains are analogue; computers are digital
\item The brain uses content-addressable memory
\item The brain is a massively parallel machine computers are modular and serial
\item Processing speed is not fixed in the brain; there is no system clock
\item Short-term memory is not like RAM
\item No hardware/software distinction can be made with respect to the brain or mind
\item Synapses are far more complex than electrical logic gates
\item Unlike computers, processing and memory are performed by the same components in the brain
\item The brain is a self-organising system
\item Brains have bodies
\item	The brain is much, much bigger than any [current] computer
\end{enumerate}

Why Minds Are Not Like Computers \autocite{Schulman2009}
Software – Hardware == Mind – Brain ??? analogy

"The power of the computer derives not from its ability to perform complex operations, but from its ability to perform many simple operations very quickly."

Layers of abstraction in computers:\\
1.	user interface\\
2.	high level programming language\\
3.	machine language\\
4.	proessor microarchitecture\\
5.	Boolean logic gates\\
6.	transistors\\

layers of abstraction in brain:\\
1.	personality?\\
2.	Thinking?\\
3.	Chemical /electrical signals/activity?\\
4.	Divided Brain regions/structure\\
5.	Neurons\\
6.	Dendrites (input) and axons (output)?\\


Computers are faster and better than humans in many tasks already.

\begin{quote}
"The weaknesses of the computational approach include its assumption that cognition can be reduced to mathematics and the difficulty of including noncognitive factors in creativity." \autocite[p.457]{Mayer1999}
\end{quote}
